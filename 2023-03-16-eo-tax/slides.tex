\PassOptionsToPackage{unicode=true}{hyperref} % options for packages loaded elsewhere
\PassOptionsToPackage{hyphens}{url}
%
\documentclass[ignorenonframetext,aspectratio=169,12pt]{beamer}
\usepackage{pgfpages}
\setbeamertemplate{caption}[numbered]
\setbeamertemplate{caption label separator}{: }
\setbeamercolor{caption name}{fg=normal text.fg}
\beamertemplatenavigationsymbolsempty
\usepackage{lmodern}
\usepackage{amssymb,amsmath}
\usepackage{ifxetex,ifluatex}
\usepackage{fixltx2e} % provides \textsubscript
\ifnum 0\ifxetex 1\fi\ifluatex 1\fi=0 % if pdftex
  \usepackage[T1]{fontenc}
  \usepackage[utf8]{inputenc}
  \usepackage{textcomp} % provides euro and other symbols
\else % if luatex or xelatex
  \usepackage{unicode-math}
  \defaultfontfeatures{Ligatures=TeX,Scale=MatchLowercase}
\fi
% use upquote if available, for straight quotes in verbatim environments
\IfFileExists{upquote.sty}{\usepackage{upquote}}{}
% use microtype if available
\IfFileExists{microtype.sty}{%
\usepackage[]{microtype}
\UseMicrotypeSet[protrusion]{basicmath} % disable protrusion for tt fonts
}{}
\IfFileExists{parskip.sty}{%
\usepackage{parskip}
}{% else
\setlength{\parindent}{0pt}
\setlength{\parskip}{6pt plus 2pt minus 1pt}
}
\usepackage{hyperref}
\hypersetup{
            pdfborder={0 0 0},
            breaklinks=true}
\urlstyle{same}  % don't use monospace font for urls
\newif\ifbibliography
% Prevent slide breaks in the middle of a paragraph:
\widowpenalties 1 10000
\raggedbottom
\setbeamertemplate{part page}{
\centering
\begin{beamercolorbox}[sep=16pt,center]{part title}
  \usebeamerfont{part title}\insertpart\par
\end{beamercolorbox}
}
\setbeamertemplate{section page}{
\centering
\begin{beamercolorbox}[sep=12pt,center]{part title}
  \usebeamerfont{section title}\insertsection\par
\end{beamercolorbox}
}
\setbeamertemplate{subsection page}{
\centering
\begin{beamercolorbox}[sep=8pt,center]{part title}
  \usebeamerfont{subsection title}\insertsubsection\par
\end{beamercolorbox}
}
\AtBeginPart{
  \frame{\partpage}
}
\AtBeginSection{
  \ifbibliography
  \else
    \frame{\sectionpage}
  \fi
}
\AtBeginSubsection{
  \frame{\subsectionpage}
}
\setlength{\emergencystretch}{3em}  % prevent overfull lines
\providecommand{\tightlist}{%
  \setlength{\itemsep}{0pt}\setlength{\parskip}{0pt}}
\setcounter{secnumdepth}{0}

% set default figure placement to htbp
\makeatletter
\def\fps@figure{htbp}
\makeatother

\DeclareUnicodeCharacter{00A0}{~}
\DeclareUnicodeCharacter{03B4}{$\delta$}
\DeclareUnicodeCharacter{03B5}{$\varepsilon$}
\DeclareUnicodeCharacter{03C9}{$\omega$}
\DeclareUnicodeCharacter{2124}{\mathbb{Z}}
\DeclareUnicodeCharacter{2193}{$\downarrow$}
\DeclareUnicodeCharacter{2208}{$\in$}
\DeclareUnicodeCharacter{2209}{$\notin$}
\DeclareUnicodeCharacter{220B}{$\ni$}
\DeclareUnicodeCharacter{2227}{$\wedge$}
\DeclareUnicodeCharacter{2228}{$\vee$}
\DeclareUnicodeCharacter{2234}{$\therefore$}
\DeclareUnicodeCharacter{2264}{$\leq$}
\DeclareUnicodeCharacter{2265}{$\geq$}
\DeclareUnicodeCharacter{2605}{$\star$}
\DeclareUnicodeCharacter{1D53D}{\mathbb{F}}


\hypersetup{colorlinks,linkcolor=,urlcolor=purple}
\setbeamertemplate{navigation symbols}{}
\usefonttheme[onlymath]{serif}

\setbeamercolor{footnote mark}{fg=gray}
\setbeamerfont{footnote}{size=\tiny}
\usepackage{color}
\usepackage[normalem]{ulem}
\usepackage{listings}
\lstset{
    basicstyle=\ttfamily\normalsize,
    keywordstyle=\color{blue}\bfseries,
    commentstyle=\color[rgb]{0,0.5,0}\bfseries\em,
    stringstyle=\color{red}\bfseries\em,
    escapeinside={(*}{*)}
}

\title{\bf A tax combinator library could be useful for {\em Rules as Code}}
\providecommand{\subtitle}[1]{}
\subtitle{\footnotesize {\em or:} surely you didn't think you'd escape without hearing me talk about Haskell?}
\author{{\bf Fraser Tweedale}\\
    \texttt{@hackuador@functional.cafe}}
\date{March 16, 2023}

\begin{document}
\frame{\titlepage}

\begin{frame}[fragile]{Rates Rebate Act 1973 (NZ) s 3(1)}
\footnotesize
\begin{verbatim}

A ratepayer who, at the commencement of a rating year, was the
ratepayer of a residential property is entitled, on application
in that year, to a rebate of—

(a) so much of the rates payable for that rating year in respect of
    the property as represents—
    (i) two-thirds of the amount by which those rates exceed $160,
        reduced by—
    (ii) $1 for each $8 by which the ratepayer’s income for the
         preceding tax year exceeded $28,080, that last-mentioned
         amount being increased by $500 in respect of each person who
         was a dependant of the ratepayer at the commencement of the
         rating year in respect of which the application is made; or
(b) $700,—

whichever amount is smaller.

\end{verbatim}

\end{frame}

\begin{frame}[fragile]{}

\begin{lstlisting}[basicstyle=\ttfamily\footnotesize]
incomeThreshold :: Natural -> Money Rational
incomeThreshold numDependants =
  Money (28080 + fromIntegral numDependants * 500)

reduction :: Natural -> Tax (Money Rational) (Money Rational)
reduction numDependants =
  (*\colorbox{yellow}{above}*) (incomeThreshold numDependants) (-1/8)

rebate :: Natural -> Money Rational
       -> Tax (Money Rational) (Money Rational)
rebate numDependants income =
  (*\colorbox{yellow}{lesserOf}*)
    ((*\colorbox{yellow}{lump}*) (Money 700))
    ((*\colorbox{yellow}{greaterOf}*)
      ((*\colorbox{yellow}{lump}*) (Money 0))
      ((*\colorbox{yellow}{above}*) (Money 160) (2/3)
       <> (*\colorbox{yellow}{lump}*) ((*\colorbox{yellow}{getTax}*) (reduction numDependants) income)))
\end{lstlisting}

\end{frame}

\begin{frame}[plain]{}
\textbf{\em Combinators}:
\begin{itemize}
  \item functions that \textbf{\em compose} smaller computations into
    a more complex computation\ldots{}
  \item that (typically) follow some \textbf{\em laws}\ldots{}
  \item such that our ability to \textbf{\em reason} about the computation can
    scale as it grows.
\end{itemize}
\end{frame}

\begin{frame}[fragile]
\center

\begin{lstlisting}
medicareLevy loThreshold =
  (*\colorbox{yellow}{lesserOf}*) ((*\colorbox{yellow}{above}*) loThreshold 0.1) ((*\colorbox{yellow}{flat}*) 0.02)
\end{lstlisting}

\end{frame}

\begin{frame}[fragile]
\center

\begin{lstlisting}
lowIncomeTaxOffset =
  (*\colorbox{yellow}{limit}*) (Money 0)
    ( (*\colorbox{yellow}{lump}*) (Money (-445))
      <> (*\colorbox{yellow}{above}*) (Money 37000) 0.015
    )
\end{lstlisting}

\end{frame}

\begin{frame}[fragile]
\center

\begin{lstlisting}
individualIncomeTax = (*\colorbox{yellow}{marginal'}*)
  [ ( 18200, 0.19         )
  , ( 45000, 0.325 - 0.19 )
  , (120000, 0.37  - 0.325)
  , (180000, 0.45  - 0.37 ) ]
\end{lstlisting}

\end{frame}


\begin{frame}[fragile]
\center

\begin{lstlisting}[language=Haskell]
data Tax b a = Tax { getTax :: (*\colorbox{yellow}{b -> a}*) }
  deriving ((*\colorbox{yellow}{Semigroup, Monoid, Functor, Profunctor}*))
\end{lstlisting}

\end{frame}

\begin{frame}[fragile]
\center

\begin{lstlisting}
allTheTaxes =
  individualIncomeTax
  (*\colorbox{yellow}{<>}*) medicareLevy (Money 23226)
  (*\colorbox{yellow}{<>}*) lowIncomeTaxOffset

weeklyWithholding  = (*\colorbox{yellow}{dimap}*) ($* 52) ($/ 52) allTheTaxes
fnlyWithholding    = (*\colorbox{yellow}{dimap}*) ($* 26) ($/ 26) allTheTaxes
monthlyWithholding = (*\colorbox{yellow}{dimap}*) ($* 12) ($/ 12) allTheTaxes
\end{lstlisting}

\end{frame}

\begin{frame}[plain]
\Large
\center
\ttfamily
\url{https://hackage.haskell.org/package/tax}\\
\bigskip
\url{https://github.com/frasertweedale/hs-tax}\\
\bigskip
blog post: \url{https://is.gd/tax_combinators}\\
\bigskip
\bigskip
useful for "Rules as Code" calculations relating to tax, levies,
rates, rebates, payroll...

\end{frame}



\begin{frame}[plain]
\Large
\center
\ttfamily
What do /I/ use it for?\\
\bigskip
hs-tax-ato - personal income tax library\\
\bigskip
\url{https://hackage.haskell.org/package/tax-ato}\\
\bigskip
\url{https://github.com/frasertweedale/hs-tax-ato}

\end{frame}

\begin{frame}{hs-tax-ato: stuff that's implemented}
\begin{itemize}
  \item individual income tax, employee share schemes
  \item Medicare levy, Medicare levy surcharge
  \item HELP (HECS/SFSS student loans)
  \item deductions
  \item offsets: low income, LMITO, spouse contribution
  \item private health insurance rebate adjustments
  \item dividends, CGT, foreign income, foreign income tax offset
  \item rates/rules for previous financial years (back to 2017)
\end{itemize}
\end{frame}

\begin{frame}{hs-tax-ato: stuff that's NOT implemented}
\begin{itemize}
  \item PAYG instalments ({\em coming soon})
  \item some adjustments/variations based on family income or dependents
  \item partnership/trust/personal services income
  \item super income streams and lump payments
  \item some "grandfathering" rules (e.g. CGT indexation method)
  \item lots of other esoteric (to me) features and quirks
\end{itemize}
\end{frame}

\begin{frame}{hs-tax-ato: known issues}
\begin{itemize}
  \item needs updates for 2022--23 FY ({\em coming soon})
  \item missing some rounding steps
  \item lack of usage examples
\end{itemize}

\end{frame}
\begin{frame}[plain]
\centering
\Large
  Maybe these libraries will be useful or interesting to someone...
\end{frame}


% END MATTER

\begin{frame}[plain]
\begin{columns}

  \begin{column}{.6\textwidth}

    \setlength{\parskip}{.5em}

    { \centering

    \input{cc-by-ARTIFACT.pdf_tex}

    \copyright~2023  Fraser Tweedale

    { \scriptsize
    Except where otherwise noted this work is licensed under
    }
    { \footnotesize
    \textbf{http://creativecommons.org/licenses/by/4.0/}
    }

    }

    \begin{description}
      \item[Code] \url{https://github.com/frasertweedale}
      \item[Blog] \href{https://frasertweedale.github.io/blog-fp/}{frasertweedale.github.io/blog-fp}
      \item[Fediverse] \href{https://functional.cafe/@hackuador}{@hackuador@functional.cafe}
    \end{description}
  \end{column}

\end{columns}
\end{frame}

\end{document}

\PassOptionsToPackage{unicode=true}{hyperref} % options for packages loaded elsewhere
\PassOptionsToPackage{hyphens}{url}
%
\documentclass[ignorenonframetext,aspectratio=169,12pt]{beamer}
\usepackage{pgfpages}
\setbeamertemplate{caption}[numbered]
\setbeamertemplate{caption label separator}{: }
\setbeamercolor{caption name}{fg=normal text.fg}
\beamertemplatenavigationsymbolsempty
\usepackage{lmodern}
\usepackage{amssymb,amsmath}
\usepackage{ifxetex,ifluatex}
\usepackage{fixltx2e} % provides \textsubscript
\ifnum 0\ifxetex 1\fi\ifluatex 1\fi=0 % if pdftex
  \usepackage[T1]{fontenc}
  \usepackage[utf8]{inputenc}
  \usepackage{textcomp} % provides euro and other symbols
\else % if luatex or xelatex
  \usepackage{unicode-math}
  \defaultfontfeatures{Ligatures=TeX,Scale=MatchLowercase}
\fi
% use upquote if available, for straight quotes in verbatim environments
\IfFileExists{upquote.sty}{\usepackage{upquote}}{}
% use microtype if available
\IfFileExists{microtype.sty}{%
\usepackage[]{microtype}
\UseMicrotypeSet[protrusion]{basicmath} % disable protrusion for tt fonts
}{}
\IfFileExists{parskip.sty}{%
\usepackage{parskip}
}{% else
\setlength{\parindent}{0pt}
\setlength{\parskip}{6pt plus 2pt minus 1pt}
}
\usepackage{hyperref}
\hypersetup{
            pdfborder={0 0 0},
            breaklinks=true}
\urlstyle{same}  % don't use monospace font for urls
\newif\ifbibliography
% Prevent slide breaks in the middle of a paragraph:
\widowpenalties 1 10000
\raggedbottom
\setbeamertemplate{part page}{
\centering
\begin{beamercolorbox}[sep=16pt,center]{part title}
  \usebeamerfont{part title}\insertpart\par
\end{beamercolorbox}
}
\setbeamertemplate{section page}{
\centering
\begin{beamercolorbox}[sep=12pt,center]{part title}
  \usebeamerfont{section title}\insertsection\par
\end{beamercolorbox}
}
\setbeamertemplate{subsection page}{
\centering
\begin{beamercolorbox}[sep=8pt,center]{part title}
  \usebeamerfont{subsection title}\insertsubsection\par
\end{beamercolorbox}
}
\AtBeginPart{
  \frame{\partpage}
}
\AtBeginSection{
  \ifbibliography
  \else
    \frame{\sectionpage}
  \fi
}
\AtBeginSubsection{
  \frame{\subsectionpage}
}
\setlength{\emergencystretch}{3em}  % prevent overfull lines
\providecommand{\tightlist}{%
  \setlength{\itemsep}{0pt}\setlength{\parskip}{0pt}}
\setcounter{secnumdepth}{0}

% set default figure placement to htbp
\makeatletter
\def\fps@figure{htbp}
\makeatother

\DeclareUnicodeCharacter{00A0}{~}
\DeclareUnicodeCharacter{03B4}{$\delta$}
\DeclareUnicodeCharacter{03B5}{$\varepsilon$}
\DeclareUnicodeCharacter{03C9}{$\omega$}
\DeclareUnicodeCharacter{2124}{\mathbb{Z}}
\DeclareUnicodeCharacter{2193}{$\downarrow$}
\DeclareUnicodeCharacter{2208}{$\in$}
\DeclareUnicodeCharacter{2209}{$\notin$}
\DeclareUnicodeCharacter{220B}{$\ni$}
\DeclareUnicodeCharacter{2227}{$\wedge$}
\DeclareUnicodeCharacter{2228}{$\vee$}
\DeclareUnicodeCharacter{2234}{$\therefore$}
\DeclareUnicodeCharacter{2264}{$\leq$}
\DeclareUnicodeCharacter{2265}{$\geq$}
\DeclareUnicodeCharacter{2605}{$\star$}
\DeclareUnicodeCharacter{1D53D}{\mathbb{F}}

% Scale images if necessary, so that they will not overflow the page
% margins by default, and it is still possible to overwrite the defaults
% using explicit options in \includegraphics[width, height, ...]{}
%\setkeys{Gin}{width=\maxwidth,height=\maxheight,keepaspectratio}
\newcommand{\includegraphicsscaled}[1]{
    \includegraphics[width=\maxwidth,height=\maxheight,keepaspectratio]{#1}
}


\hypersetup{colorlinks,linkcolor=,urlcolor=purple}
\setbeamertemplate{navigation symbols}{}
\usefonttheme[onlymath]{serif}

\setbeamercolor{footnote mark}{fg=gray}
\setbeamerfont{footnote}{size=\tiny}
\usepackage{color}
\usepackage[normalem]{ulem}
\usepackage{listings}
\lstset{
    basicstyle=\ttfamily\normalsize,
    keywordstyle=\color{blue}\bfseries,
    commentstyle=\color[rgb]{0,0.5,0}\bfseries\em,
    stringstyle=\color{red}\bfseries\em,
    escapeinside={(*}{*)}
}

\usepackage{graphicx,grffile}
\makeatletter
\def\maxwidth{\ifdim\Gin@nat@width>\linewidth\linewidth\else\Gin@nat@width\fi}
\def\maxheight{\ifdim\Gin@nat@height>\textheight\textheight\else\Gin@nat@height\fi}
\makeatother

\title{\bf The Haskell Security Response Team}
\providecommand{\subtitle}[1]{}
\subtitle{Current status and future evolution}
\author{{\bf Fraser Tweedale}\\
    \texttt{@hackuador@functional.cafe}}
\date{2024-06-06}

\begin{document}
\frame{\titlepage}

\begin{frame}{Outline}
    \begin{itemize}
        \item SRT: why / what / who
        \item Advisory database status
        \item Other high-priority work
        \item Evolving the SRT
        \item ZuriHac goals / ideas
    \end{itemize}
\end{frame}

\begin{frame}
\centering
\Large
  SRT: why / what / who
\end{frame}

\begin{frame}{SRT - motivation}
  \begin{itemize}
    \item
      \href{https://github.com/david-christiansen/tech-proposals/blob/ae8d4ef73df24cd053f3b88683456a6cbc4757de/proposals/accepted/037-advisory-db.md}
        {HF tech proposal 37}: advisory database
    \item Primary motivation: enterprise adoption
    \item Without a security response structure and artifacts,
      Haskell is a {\bf non-starter} for many companies
    \item Makes Haskell an easier choice even without
      hard regulatory/compliance requirements
    \item We should care about security of our ecosystem anyway :)
  \end{itemize}
\end{frame}

\begin{frame}{SRT - scope}
    \begin{itemize}
        \item Manage the advisory database and associated tooling
        \item Triage, assess and admit issue reports
        \item Coordinate repsonse with maintainers of affected
            packages (high-impact issues)
        \item Collaborate and respond to needs of downstream tools
            that consume our advisories
        \item Quarterly report
    \end{itemize}
\end{frame}

\begin{frame}{SRT - current members}
    \begin{itemize}
        \item Fraser Tweedale
        \item Gautier Di Folco
        \item Mihai Maruseac
        \item Tristan de Cacqueray
        \item Casey Mattingly
        \item Jose Calderon (observer/overseer)
    \end{itemize}
\end{frame}

\begin{frame}
\centering
\Large
  Advisory database
\end{frame}

\begin{frame}{Advisory database - structure}
  \begin{itemize}
    \item \url{https://github.com/haskell/security-advisories}
    \item TOML metadata + CommonMark description
    \item arranged by namespace and package/component name
    \item symlinks for multi-package advisories
    \item quirk: dates are derived from Git commit times
  \end{itemize}
\end{frame}

\begin{frame}[fragile]
\scriptsize
\begin{verbatim}
```toml
[advisory]
id = "HSEC-2023-0001"
cwe = [328, 400]
keywords = ["json", "dos", "historical"]
aliases = ["CVE-2022-3433"]

[[affected]]
package = "aeson"
cvss = "CVSS:3.1/AV:N/AC:L/PR:L/UI:N/S:U/C:N/I:N/A:H"

[[affected.versions]]
introduced = "0.4.0.0"
fixed = "2.0.1.0"
```

# Hash flooding vulnerability in aeson

*aeson* was vulnerable to hash flooding (a.k.a. hash DoS).  The
issue is a consequence of the HashMap implementation from
*unordered-containers*.  It results in a denial of service through
CPU consumption.  This technique has been used in real-world attacks
against a variety of languages, libraries and frameworks over the
years.

\end{verbatim}
\end{frame}

\begin{frame}{Advisory database - outputs}
  \begin{itemize}
    \item OSV (ingested by
      osv.dev\footnote{\url{https://osv.dev/list?ecosystem=Hackage}})
    \item HTML index: \url{https://haskell.github.io/security-advisories/}
    \item "snapshot" format designed for syncing with
      tools\footnote{\url{https://github.com/haskell/security-advisories/pull/179}}
  \end{itemize}
\end{frame}

\begin{frame}{Advisory database - libraries}

  Libraries and tools for processing advisory data are on Hackage:

  \begin{itemize}
    \item \url{https://hackage.haskell.org/package/cvss}
    \item \url{https://hackage.haskell.org/package/osv}
    \item \url{https://hackage.haskell.org/package/hsec-core}
    \item \url{https://hackage.haskell.org/package/hsec-tools}
  \end{itemize}

  Expect churn as more consumers/users arrive, give feedback.

  CWE\footnote{{\em Common Weakness Enumeration}---\url{https://cwe.mitre.org/}} library is coming.

\end{frame}


\begin{frame}{Advisory database - low growth}
  \begin{itemize}
    \item SRT advisory triage/assess/add workload is {\bf very low}.
      Why?
    \item Submission process is too hard? (improve the tools)
    \item Haskell just has fewer security bugs? (not
      significantly, IMO)
    \item People don't know about it? (increase visibility)
    \item People don't care?
  \end{itemize}
\end{frame}

\begin{frame}
\centering
\Large
  What is missing?
\end{frame}

\begin{frame}{cabal-audit}
  \begin{itemize}
    \item Scans the build plan for vulnerable dependencies
    \item The author, {\bf MangoIV}, is here!
    \item Long-term goal: integrated with {\tt cabal-install} (as
      plugin)
      \begin{itemize}
        \item Ship via GHCUp?
      \end{itemize}
    \item \url{https://github.com/MangoIV/cabal-audit}
  \end{itemize}
\end{frame}

\begin{frame}[plain]
  \begin{center}
    \includegraphicsscaled{cabal-audit-screenshot.png}
  \end{center}
\end{frame}

\begin{frame}{{\bf Hackage} - integrate with advisory db}
  \begin{itemize}
    \item Show info about vulnerable versions
    \item Show info about (potentially) vulnerable deps
    \item Add "how to report" info / helpers
    \item Flora.pm might want to do these things too
  \end{itemize}
\end{frame}

\begin{frame}{{\bf Hackage} - other security enhancements}
  \begin{itemize}
    \item password / token storage improvements
    \item 2FA (TOTP for a start)
  \end{itemize}
\end{frame}

\begin{frame}{SRT tooling}
  \begin{itemize}
    \item {\bf GitHub bot} to help define / review advisories
      \begin{itemize}
        \item explain CVSS, CWE values; suggest keywords; etc
      \end{itemize}
    \item {\bf Web form} for advisory submission
      \begin{itemize}
        \item \ldots{}or other ways to make it easier
      \end{itemize}
  \end{itemize}
\end{frame}

\begin{frame}{Exploitability information}
  \begin{itemize}
    \item As audit tooling matures, we must suppress false positives
      \begin{itemize}
        \item e.g. HSEC-2023-0007 {\tt readFloat} memory exhaustion
          in {\em \textbf{base}}
      \end{itemize}
      
    \item VEX - {\em Vulnerability Exploitability eXchange}
      \begin{itemize}
        \item statements that an issue is(n't) exploitable in the
          dependent
        \item {\bf data model} by CISA.gov
        \item implementations:
          \href{https://github.com/openvex/spec}{OpenVEX},
          \href{https://spdx.dev/capturing-software-vulnerability-data-in-spdx-3-0/}{SPDX 3.0},
          \href{https://www.oasis-open.org/2022/11/21/new-version-of-csaf-standard/}{OASIS CSAF 2.0},
          CycloneDX 
      \end{itemize}
    \item We don't have to use VEX {\em per se}
  \end{itemize}
\end{frame}

\begin{frame}{VEX statements - generation}
  \begin{itemize}
    \item Written by human
    \item Generated by machine (call analysis)
      \begin{itemize}
        \item Tristan's experiment: \url{https://github.com/TristanCacqueray/cabal-audit}
        \item typeclass methods seem to be the tricky part
        \item relies on declaration of affected
          functions/symbols in the advisory
      \end{itemize}
  \end{itemize}
\end{frame}

\begin{frame}{VEX statements - distribution}
  \begin{itemize}
    \item Cabal package description ({\bf dependent})
      \begin{itemize}
        \item supplied by maintainer, or Hackage trustees
        \item distributed in Hackage snapshots
        \item metadata revisions $\to$ new version not required to
          update VEX statements
      \end{itemize}
    \item Advisory DB as a VEX clearing-house
      \begin{itemize}
        \item supplied by SRT, or community with SRT review
        \item distributed in Advisory DB snapshots
      \end{itemize}
    \item Both?
      \begin{itemize}
        \item requires conflict resolution (preferred source)
      \end{itemize}
  \end{itemize}
\end{frame}

\begin{frame}{Dependency graph analysis}
  \begin{itemize}
    \item Tools to analyse the dependency graph (of a single project
      or whole ecosystem) are increasingly important
    \item That xkcd\footnote{\url{https://xkcd.com/2347/}}
    \item Identify the {\bf load-bearing projects / juicy targets}?
      \begin{itemize}
        \item Are they maintained?  Sustainably?
        \item What are the main risks?
      \end{itemize}
    \item Query projects exposed to {\bf external risks}
      \begin{itemize}
        \item cbits?  vendored/bundled code?  out of date?
        \item using external libraries?
      \end{itemize}
  \end{itemize}
\end{frame}

\begin{frame}{Dependency graph analysis - Open Source Insights}
  \begin{itemize}
    \item A fair bit of this tooling exists in
      {\em Open Source Insights}
      \begin{itemize}
        \item \url{https://deps.dev}; Google project
        \item Web, visualisations, API, BigQuery
      \end{itemize}
    \item Haskell is not supported yet
    \item Development is not public (currently)
    \item I have reached out to find out more and offer support
    \item \href{https://hackage.haskell.org/package/acme-everything}
      {\em acme-everything} becomes useful?
  \end{itemize}
\end{frame}

\begin{frame}{But wait there's more\ldots{}}
  \begin{itemize}
    \item SBOM artifacts? (e.g. \href{https://spdx.dev/}{SPDX})
    \item Add Haskell call analysis support to
      \href{https://google.github.io/osv-scanner}{osv-scanner}?
    \item Increase issue discovery efforts
      \begin{itemize}
        \item \href{https://github.com/google/oss-fuzz}{OSS-Fuzz} support?
      \end{itemize}
    \item \href{https://www.bestpractices.dev/en/criteria/0}{OpenSSF Best Practices} checking?
    \item {\bf Verified crypto} libs
      \begin{itemize}
        \item \ldots{}and other compliance things that matter in various
          sectors
      \end{itemize}
  \end{itemize}
\end{frame}

\begin{frame}{Software Bill of Materials (SBOM)}
  \begin{itemize}
    \item Blog post published yesterday:
      \begin{itemize}
        \item {\em SBOM Generation and Vulnerability Monitoring for
          the Hackage/Haskell Ecosystem}\footnote{\url{https://www.timesys.com/security/sbom-generation-and-vulnerability-monitoring-for-the-hackage-haskell-ecosystem/}}---Timesys (cybersecurity
          vendor)
        \item Cabal freeze file $\to$ CycloneDX SBOM via
          {\em syft}\footnote{\url{https://github.com/anchore/syft}}
      \end{itemize}
    \item Build SBOM generation into {\tt cabal-install}?  As
      plugin?
  \end{itemize}
\end{frame}

\begin{frame}
\centering
\Large
  Evolution of the SRT
\end{frame}

\begin{frame}{SRT scope change}
  \begin{itemize}
    \item Workload w.r.t. current charter is low
    \item Much to do that's {\em not} in the charter
    \item Therefore: {\bf expand the charter} and grow the team

    \item Gather feedback/ideas this week about:
        \begin{itemize}
          \item how scope of SRT should change
          \item topology (sub-teams? separate efforts? how many
            people?)
        \end{itemize}

  \end{itemize}
\end{frame}

\begin{frame}{SRT nominations}
  \begin{itemize}
    \item Soon: call for nominations for new SRT members
    \item Number of people and responsibilities depends on feedback
    \item Casey is retiring from SRT.  {\bf Thank you} for your efforts!
  \end{itemize}
\end{frame}


\begin{frame}
\centering
\Large
  Ecosystem Workshop / ZuriHac goals
\end{frame}

\begin{frame}{Ecosystem Workshop / ZuriHac goals}
  \begin{itemize}
    \item Collaborate with anyone on anything that improves
      Haskell security posture
    \item Discussions about SRT evolution
    \item I will {\em default} to working on Hackage
    \item Gautier will work on advisory db snapshots
    \item MangoIV will work on cabal-audit
    \item Beginner-friendly tasks:
      \begin{itemize}
        \item {\em cvss}/{\em osv}/{\em hsec-core} CVSS 4.0 support
        \item Haskell.org security page -
          \url{https://github.com/haskell-infra/www.haskell.org/issues/293}
      \end{itemize}
  \end{itemize}
\end{frame}


% END MATTER

\begin{frame}[plain]
\begin{columns}

  \begin{column}{.6\textwidth}

    \setlength{\parskip}{.5em}

    { \centering

    \input{cc-by-ARTIFACT.pdf_tex}

    \copyright~2024  Fraser Tweedale

    { \scriptsize
    Except where otherwise noted this work is licensed under
    }
    { \footnotesize
    \textbf{http://creativecommons.org/licenses/by/4.0/}
    }

    }

    \begin{description}
      \small
      \item[SRT code] \href{https://github.com/haskell/security-advisories}{github.com/haskell/security-advisories}
      \item[My blog] \href{https://frasertweedale.github.io/blog-fp/}{frasertweedale.github.io/blog-fp}
      \item[Fediverse] \href{https://functional.cafe/@hackuador}{@hackuador@functional.cafe}
    \end{description}
  \end{column}

\end{columns}
\end{frame}

\end{document}

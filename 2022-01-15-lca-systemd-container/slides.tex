\PassOptionsToPackage{unicode=true}{hyperref} % options for packages loaded elsewhere
\PassOptionsToPackage{hyphens}{url}
%
\documentclass[ignorenonframetext,aspectratio=169,12pt]{beamer}
\usepackage{pgfpages}
\setbeamertemplate{caption}[numbered]
\setbeamertemplate{caption label separator}{: }
\setbeamercolor{caption name}{fg=normal text.fg}
\beamertemplatenavigationsymbolsempty
\usepackage{lmodern}
\usepackage{amssymb,amsmath}
\usepackage{ifxetex,ifluatex}
\usepackage{fixltx2e} % provides \textsubscript
\ifnum 0\ifxetex 1\fi\ifluatex 1\fi=0 % if pdftex
  \usepackage[T1]{fontenc}
  \usepackage[utf8]{inputenc}
  \usepackage{textcomp} % provides euro and other symbols
\else % if luatex or xelatex
  \usepackage{unicode-math}
  \defaultfontfeatures{Ligatures=TeX,Scale=MatchLowercase}
\fi
% use upquote if available, for straight quotes in verbatim environments
\IfFileExists{upquote.sty}{\usepackage{upquote}}{}
% use microtype if available
\IfFileExists{microtype.sty}{%
\usepackage[]{microtype}
\UseMicrotypeSet[protrusion]{basicmath} % disable protrusion for tt fonts
}{}
\IfFileExists{parskip.sty}{%
\usepackage{parskip}
}{% else
\setlength{\parindent}{0pt}
\setlength{\parskip}{6pt plus 2pt minus 1pt}
}
\usepackage{hyperref}
\hypersetup{
            pdfborder={0 0 0},
            breaklinks=true}
\urlstyle{same}  % don't use monospace font for urls
\newif\ifbibliography
% Prevent slide breaks in the middle of a paragraph:
\widowpenalties 1 10000
\raggedbottom
\setbeamertemplate{part page}{
\centering
\begin{beamercolorbox}[sep=16pt,center]{part title}
  \usebeamerfont{part title}\insertpart\par
\end{beamercolorbox}
}
\setbeamertemplate{section page}{
\centering
\begin{beamercolorbox}[sep=12pt,center]{part title}
  \usebeamerfont{section title}\insertsection\par
\end{beamercolorbox}
}
\setbeamertemplate{subsection page}{
\centering
\begin{beamercolorbox}[sep=8pt,center]{part title}
  \usebeamerfont{subsection title}\insertsubsection\par
\end{beamercolorbox}
}
\AtBeginPart{
  \frame{\partpage}
}
\AtBeginSection{
  \ifbibliography
  \else
    \frame{\sectionpage}
  \fi
}
\AtBeginSubsection{
  \frame{\subsectionpage}
}
\setlength{\emergencystretch}{3em}  % prevent overfull lines
\providecommand{\tightlist}{%
  \setlength{\itemsep}{0pt}\setlength{\parskip}{0pt}}
\setcounter{secnumdepth}{0}

% set default figure placement to htbp
\makeatletter
\def\fps@figure{htbp}
\makeatother

\DeclareUnicodeCharacter{00A0}{~}
\DeclareUnicodeCharacter{03B4}{$\delta$}
\DeclareUnicodeCharacter{03B5}{$\varepsilon$}
\DeclareUnicodeCharacter{03C9}{$\omega$}
\DeclareUnicodeCharacter{2124}{\mathbb{Z}}
\DeclareUnicodeCharacter{2193}{$\downarrow$}
\DeclareUnicodeCharacter{2208}{$\in$}
\DeclareUnicodeCharacter{2209}{$\notin$}
\DeclareUnicodeCharacter{220B}{$\ni$}
\DeclareUnicodeCharacter{2227}{$\wedge$}
\DeclareUnicodeCharacter{2228}{$\vee$}
\DeclareUnicodeCharacter{2234}{$\therefore$}
\DeclareUnicodeCharacter{2264}{$\leq$}
\DeclareUnicodeCharacter{2265}{$\geq$}
\DeclareUnicodeCharacter{2605}{$\star$}
\DeclareUnicodeCharacter{1D53D}{\mathbb{F}}


\hypersetup{colorlinks,linkcolor=,urlcolor=purple}
\setbeamertemplate{navigation symbols}{}
\usefonttheme[onlymath]{serif}

\setbeamercolor{footnote mark}{fg=gray}
\setbeamerfont{footnote}{size=\tiny}
\usepackage{color}
\usepackage[normalem]{ulem}
\usepackage{listings}
\lstset{
    basicstyle=\ttfamily\normalsize,
    keywordstyle=\color{blue}\bfseries,
    commentstyle=\color[rgb]{0,0.5,0}\bfseries\em,
    stringstyle=\color{red}\bfseries\em,
    escapeinside={(*}{*)}
}

\title{\bf send in the {\tt chown}s}
\providecommand{\subtitle}[1]{}
\subtitle{\bf systemd containers on OpenShift}
\author{{\bf Fraser Tweedale}\\
    \texttt{@hackuador}\\
    \bigskip
    \def\svgwidth{4cm}
    \input{Logo-RedHat-A-Color-RGB-ARTIFACT.pdf_tex}}
\date{January 15, 2022}

\begin{document}
\frame{\titlepage}

\begin{frame}{Preliminaries}
\protect\hypertarget{preliminaries}{}
\begin{itemize}
    \item CC-BY 4.0, except where otherwise noted
    \item Slides:
      \href{https://speakerdeck.com/frasertweedale/systemd-containers-on-openshift}{speakerdeck.com/frasertweedale/systemd-containers-on-openshift}
    \item I will be available in the chatroom following the presentation
\end{itemize}
\end{frame}


\begin{frame}{Agenda}
\protect\hypertarget{agenda}{}

\begin{itemize}
    \item Containers and container standards
    \item Kubernetes and OpenShift
    \item FreeIPA: overview and use cases
    \item systemd-based workloads on Kubernetes/OpenShift
\end{itemize}

\end{frame}


\begin{frame}{What is a container?}
\protect\hypertarget{container-definition}{}

\begin{itemize}
    \item An process isolation and confinement {\em abstraction}
    \item Most commonly: OS-level virtualisation (shared kernel)
        \begin{itemize}
        \item e.g. FreeBSD jails, Solaris zones
        \end{itemize}
    \item Container \emph{\textbf{image}} defines filesystem contents
\end{itemize}

\end{frame}


\begin{frame}{Containers on linux}
\protect\hypertarget{container-linux}{}

\begin{itemize}
    \item {\tt namespaces}: pid, mount, network, cgroup, \ldots{}
    \item (maybe) SELinux/AppArmor
    \item (maybe) restricted \texttt{capabilities(7)} or
        \texttt{seccomp(2)} profile
\end{itemize}

\end{frame}

\begin{frame}{Container standards}
\protect\hypertarget{container-standards}{}
\begin{itemize}
    \item {\em Open Container Initiative (OCI)\footnote{\url{https://opencontainers.org}}}
    \item {\bf Runtime Specification}\footnote{\url{https://github.com/opencontainers/runtime-spec}} - low level runtime interface
    \begin{itemize}
        \item Linux, Solaris, Windows, VMs, \ldots{}
        \item Implementations\footnote{\url{https://github.com/opencontainers/runtime-spec/blob/main/implementations.md}}:
          {\bf runc}\footnote{\url{https://github.com/opencontainers/runc}} (reference implementation),
          crun\footnote{\url{https://github.com/containers/crun}},
          Kata Containers\footnote{\url{https://katacontainers.io/}}
    \end{itemize}
\end{itemize}
\end{frame}

\begin{frame}{OCI Runtime Specification}
\protect\hypertarget{oci-runtime-spec}{}
\begin{itemize}

\item JSON configuration (example\footnote{\url{https://github.com/opencontainers/runtime-spec/blob/main/config.md\#configuration-schema-example}})

\item mounts, process and environment, lifecycle hooks, \ldots{}

\item Linux-specific: capabilities, namespaces, cgroup, sysctls,
      {\tt seccomp} profile

\end{itemize}

\end{frame}



\begin{frame}[plain]
\centering
\huge Kubernetes and OpenShift
\end{frame}


\begin{frame}{Kubernetes - container orchestration}
\protect\hypertarget{kubernetes-intro}{}

\begin{itemize}
\item Abbreviation: {\bf ``k8s''}
\item A container orchestration system
\item Declarative configuration of container-based applications
\item Integration with many cloud providers
\item \url{https://kubernetes.io/}
\item \url{https://github.com/kubernetes/}
\end{itemize}

\end{frame}


\begin{frame}{Kubernetes - terminology}
\protect\hypertarget{kubernetes-terminology}{}

\begin{itemize}
\item {\bf Container}: isolated/confined process [tree]
\item {\bf Pod}: group (1+) of related Containers (e.g. HTTP app + database)
\item {\bf Namespace}: object and auth[nz] scope, such as for a team/project
\item {\bf Node}: a machine in the cluster; where Pods are executed
\end{itemize}

\end{frame}

\begin{frame}{Kubernetes - more terminology}
\protect\hypertarget{kubernetes-terminology-2}{}
\begin{itemize}

\item {\bf
  Kubelet}\footnote{\url{https://kubernetes.io/docs/reference/command-line-tools-reference/kubelet/}}:
  agent that executes Pods on Nodes

\item {\bf Sandbox}: isolation/confinement mechanism(s); one per Pod
\item {\bf Container Runtime Interface
  (CRI)}\footnote{\url{https://kubernetes.io/docs/concepts/architecture/cri/}}:
  interface used by Kubelet to create/start/stop/destroy Sandboxes and
  Containers
  \begin{itemize}
  \item CRI-O\footnote{\url{https://cri-o.io/}}
  \item containerd\footnote{\url{https://containerd.io/}}
  \end{itemize}
\end{itemize}
\end{frame}

\begin{frame}{Kubernetes - Container Runtime Interface}
\begin{center}
\def\svgwidth{\textwidth}
\input{figure-cri-ARTIFACT.pdf_tex}
\end{center}
\end{frame}

\begin{frame}{Kubernetes - Container Runtime Interface - CRI-O}
\begin{center}
\def\svgwidth{\textwidth}
\input{figure-cri-crio-oci-ARTIFACT.pdf_tex}
\end{center}
\end{frame}

\begin{frame}{Kubernetes - Container Runtime Interface - CRI-O + runc}
\begin{center}
\def\svgwidth{\textwidth}
\input{figure-cri-crio-runc-ARTIFACT.pdf_tex}
\end{center}
\end{frame}

\begin{frame}[fragile]{Kubernetes - Pod definition}
\protect\hypertarget{kubernetes-pod-spec}{}
\begin{lstlisting}
apiVersion: v1
(*\colorbox{yellow}{kind: Pod}*)
metadata:
  name: fedora
  labels:
    app: fedora
spec:
  containers:
  - name: fedora
    (*\colorbox{yellow}{image}*): registry.fedoraproject.org/fedora:35-x86_64
    (*\colorbox{yellow}{command}*): ["sleep", "3600"]
    (*\colorbox{yellow}{env}*):
    - name: DEBUG
      value: "1"
\end{lstlisting}
\end{frame}

\begin{frame}{OpenShift\footnote{\url{https://openshift.com/}}}
\protect\hypertarget{openshift-intro}{}

\begin{itemize}
\item a.k.a. {\em OpenShift Container Platform (OCP)}
\item An {\em enterprise-ready Kubernetes container platform}
\item Commercially supported by Red Hat
\item Community ``upstream'' distribution: OKD\footnote{\url{https://www.okd.io/}}
\item Uses {\bf CRI-O} and {\bf runc}
\item Latest stable release: 4.9
\end{itemize}

\end{frame}

\begin{frame}{OpenShift - terminology}
\protect\hypertarget{openshift-terminology}{}

\begin{itemize}

\item All existing Kubernetes terminology, plus\ldots{}

\item {\bf Project}: Extends the {\em Namespace} concept

\item {\bf Security Context Constraint (SCC)}: policy affecting
    SELinux context, {\tt seccomp} profile, {\tt capabilities}, UID

\end{itemize}

\end{frame}

\begin{frame}{OpenShift runtime environment (today)}
\protect\hypertarget{openshift-runtime-today}{}
\begin{itemize}
\item Sandboxes use SELinux, namespaces (cgroup, pid, mount, uts, network)
\item Each {\em Project} gets assigned a unique UID range
\item Containers run as a UID from that range
    \begin{itemize}
    \item Circumvent via {\tt RunAsUser} and SCCs ({\bf bad idea})
    \end{itemize}
\end{itemize}
\end{frame}


\begin{frame}[plain]
\centering
\huge FreeIPA
\end{frame}


\begin{frame}{FreeIPA}
\protect\hypertarget{freeipa-intro}{}

\begin{itemize}
\item Open Source identity management solution
\item Users, groups, services, authentication, access policies
\item 389 DS (LDAP), MIT Kerberos, Apache, Dogtag PKI, SSSD, \ldots{}
\item Part of RHEL (commercial support) and Fedora (community support)
\item \url{https://www.freeipa.org/}
\end{itemize}

\end{frame}


\begin{frame}{FreeIPA on Kubernetes/OpenShift - use cases}
\protect\hypertarget{freeipa-openshift-use-cases}{}
Identity services\ldots{}
\begin{itemize}
\item for business applications running on the cluster
\item for the cluster itself (API access, node access)
\item for an entire organisation, hosted on their OpenShift cluster
\item {\em as a service}, hosted and managed by a service provider
\end{itemize}
\end{frame}

\begin{frame}{FreeIPA container}
\protect\hypertarget{freeipa-container}{}
\begin{itemize}
\item Encapsulate the whole RHEL/Fedora-based system in a container
\item PID 1 is {\tt systemd}, which starts/manages all services
\item We call this a {\em monolithic container}
\end{itemize}
\end{frame}


\begin{frame}{Whyyyy?!}
\protect\hypertarget{freeipa-monolith-why}{}
\begin{itemize}
\item Big engineering effort to rearchitect FreeIPA to be "cloud native"
\item {\em Ongoing costs} as we support two
    different application architectures
\item If we were starting from scratch today\ldots{}
\end{itemize}
\end{frame}


\begin{frame}{FreeIPA on OpenShift - challenges}
\protect\hypertarget{freeipa-openshift-challenges}{}
\begin{itemize}
\item Unsurprisingly, there are many
\item Main areas:
    \begin{itemize}
    \item {\bf runtime}
    \item volumes and mounts
    \item
    ingress\footnote{\url{https://frasertweedale.github.io/blog-redhat/posts/2021-11-18-k8s-tcp-udp-ingress.html}}\textsuperscript{,}\footnote{\url{https://frasertweedale.github.io/blog-redhat/posts/2020-12-08-k8s-srv-limitation.html}}
    \end{itemize}
\end{itemize}
\end{frame}


\begin{frame}[plain]
\centering
\huge Challenges, workarounds and solutions
\end{frame}


\begin{frame}{Runtime - user namespaces}
\protect\hypertarget{freeipa-openshift-runtime-userns}{}
\begin{itemize}
\item systemd and other components expect to run as {\tt root} or
  other specific UID
\item Solution: {\tt user\_namespaces(7)}
  \begin{itemize}
  \item Implemented in CRI-O, since OpenShift 4.7
  \item Opt-in via Pod annotation
  \item Requires non-default cluster configuration
  \item Requires Pod to be admitted via {\tt anyuid} (or
      similar) SCC\footnote{I am working on a way to avoid this}
  \end{itemize}
\end{itemize}
\end{frame}

\begin{frame}{Runtime - user namespaces}
\protect\hypertarget{runtime-userns-figure}{}
\begin{center}
\def\svgwidth{\textwidth}
\input{figure-user-namespaces-ARTIFACT.pdf_tex}
\end{center}
\end{frame}

\begin{frame}[fragile]{Runtime - user namespaces}
\protect\hypertarget{runtime-userns-spec}{}
\begin{lstlisting}
apiVersion: v1
kind: Pod
metadata:
  name: nginx
  labels:
    app: nginx
  annotations:
    (*\colorbox{yellow}{io.openshift.builder: "true"}*)
    (*\colorbox{yellow}{io.kubernetes.cri-o.userns-mode: "auto:size=65536"}*)
spec:
  containers:
  - name: nginx
    image: quay.io/ftweedal/test-nginx:latest
    tty: true
\end{lstlisting}
\end{frame}

\begin{frame}{Runtime - user namespaces - Kubernetes support}
\protect\hypertarget{runtime-userns-kubernetes}{}
\begin{itemize}
\item KEP\footnote{Kubernetes Enhancement Proposal}-127: a long-running and ongoing discussion
\item First proposal: \url{https://github.com/kubernetes/enhancements/pull/1903}
\item Second proposal: \url{https://github.com/kubernetes/enhancements/pull/2101}
\item Current proposal: \url{https://github.com/kubernetes/enhancements/pull/3065}
\end{itemize}
\end{frame}

\begin{frame}{Runtime - cgroups}
\protect\hypertarget{openshift-cgroup}{}
\begin{itemize}

\item OpenShift creates a unique cgroup\footnote{\tt cgroups(7)} for each container

\item cgroup namespace\footnote{\tt cgroup\_namespaces(7)} makes it the ``root''
      namespace inside the container

\item cgroupfs mounts it at {\tt /sys/fs/cgroup}

\item {\tt systemd} needs write access\ldots{} but doesn't have it

\end{itemize}
\end{frame}

\begin{frame}{Runtime - cgroup ownership}
\protect\hypertarget{openshift-cgroup-ownership}{}
\begin{itemize}

\item Solution: modify runtime to {\tt chown} the cgroup to the container
      process UID

\item But first: extend OCI Runtime Spec with semantics for
      cgroup ownership\footnote{\url{https://github.com/opencontainers/runtime-spec/blob/main/config-linux.md\#cgroup-ownership}}

\item runc pull request\footnote{\url{https://github.com/opencontainers/runc/pull/3057}}
  \begin{itemize}
  \item Merged; release expected in OpenShift 4.11 or later
  \end{itemize}

\end{itemize}
\end{frame}

\begin{frame}{Runtime - OCI cgroup ownership semantics}
\protect\hypertarget{openshift-cgroup-ownership-semantics}{}
{\tt chown} container's cgroup to host UID matching the process UID in
container's user namespace, if and only if\ldots{}
\begin{itemize}
\item cgroups v2 in use, and
\item container has its own cgroup namespace, and
\item cgroupfs is mounted read/write
\end{itemize}
\end{frame}

\begin{frame}{Runtime - OCI cgroup ownership semantics}
\protect\hypertarget{openshift-cgroup-ownership-semantics-files}{}
Only the {\bf cgroup directory itself}, and the files mentioned in {\tt
/sys/kernel/cgroup/delegate}, should be {\tt chown}'d:

\begin{itemize}
\item {\tt cgroup.procs}
\item {\tt cgroup.threads}
\item {\tt cgroup.subtree\_control}
\item {\tt memory.oom.group}\footnote{depends on kernel version}
\end{itemize}

\end{frame}


\begin{frame}{Runtime - cgroups v2}
\protect\hypertarget{openshift-cgroups-v2}{}
\begin{itemize}
\item cgroups v2 is required for secure cgroup delegation
\item it works, but is not yet the default cluster configuration
\item on the roadmap
\end{itemize}
\end{frame}

\begin{frame}[fragile]{Runtime - cluster configuration (OCP 4.10) - 1/3}
\protect\hypertarget{openshift-cluster-config-1}{}
\begin{lstlisting}
apiVersion: machineconfiguration.openshift.io/v1
kind: (*\colorbox{yellow}{MachineConfig}*)
metadata:
  name: enable-cgroupv2-workers
  labels:
    machineconfiguration.openshift.io/role: worker
spec:
  (*\colorbox{yellow}{kernelArguments}*):
    - systemd.unified_cgroup_hierarchy=1
    - cgroup_no_v1="all"
    - psi=1
  ...
\end{lstlisting}
\end{frame}

\begin{frame}[fragile]{Runtime - cluster configuration (OCP 4.10) - 2/3}
\protect\hypertarget{openshift-cluster-config-2}{}
\begin{lstlisting}
  config:
    ignition:
      version: 3.1.0
    storage:
      (*\colorbox{yellow}{files}*):
      - path: (*\colorbox{yellow}{/etc/subuid}*)
        overwrite: true
        contents:
          source: data:text/plain;charset=utf-8;base64,Y29yZToxMDAwMDA6NjU1MzYKY29udGFpbmVyczoyMDAwMDA6MjY4NDM1NDU2Cg==
      - path: (*\colorbox{yellow}{/etc/subgid}*)
        overwrite: true
        contents:
          source: data:text/plain;charset=utf-8;base64,Y29yZToxMDAwMDA6NjU1MzYKY29udGFpbmVyczoyMDAwMDA6MjY4NDM1NDU2Cg==
    ...
\end{lstlisting}
\end{frame}

\begin{frame}[fragile]{Runtime - cluster configuration (OCP 4.10) - 3/3}
\protect\hypertarget{openshift-cluster-config-2}{}
\begin{lstlisting}
    (*\colorbox{yellow}{systemd}*):
      (*\colorbox{yellow}{units}*):
      - name: "rpm-overrides.service"
        enabled: true
        contents: |
          [Unit]
          Description=Install RPM overrides
          After=network-online.target rpm-ostreed.service
          [Service]
          ExecStart=/bin/sh -c 'rpm -q runc-1.0.3-992.rhaos4.10.el8.x86_64 \
            || (*\colorbox{yellow}{rpm-ostree override replace}*) --reboot https://ftweedal.fedorapeople.org/runc-1.0.3-992.rhaos4.10.el8.x86_64.rpm'
          Restart=on-failure
          [Install]
          WantedBy=multi-user.target
\end{lstlisting}
\end{frame}

% TODO main demo? (putting it all together)
\begin{frame}[plain]
\centering
\huge Demo
\end{frame}


\begin{frame}{Links / resources}
\protect\hypertarget{links}{}

\begin{itemize}
\item Project main repo: \url{https://github.com/freeipa/freeipa-openshift}
    \begin{itemize}
    \item not much here yet, watch this space
    \end{itemize}
\item runc builds: \url{https://ftweedal.fedorapeople.org/}
\item Team blogs:
    \begin{itemize}
    \item \url{https://frasertweedale.github.io/blog-redhat/tags/containers.html}
    \item \url{https://avisiedo.github.io/docs/}
    \end{itemize}
\item Demo: \url{https://www.youtube.com/watch?v=OGAVvIJwmd0}
\end{itemize}

\end{frame}


\begin{frame}{Status and future}
\protect\hypertarget{future}{}

\begin{itemize}
\item Kubernetes: user namespaces support in an ongoing discussion
\item OpenShift: systemd container in user namespace works, but experimental
\item Official support is an open question
    \begin{itemize}
    \item We are hopeful, collaborating with OpenShift project and
      product management, looking for allies
    \item But we may end up having to rearchitect FreeIPA for the cloud
    \end{itemize}
\end{itemize}

\end{frame}


\begin{frame}[plain]
\begin{columns}

  \begin{column}{.4\textwidth}
    \begin{center}
      \includegraphics[width=\textwidth]{img/elias.jpg}
    \end{center}
    \tiny Elias Wicked Ales \& Spirits
    \\ \scalebox{.9}{\url{https://www.facebook.com/wickedelias/posts/2967000120196980}}
    \\ Fair dealing for purpose of parody or satire
  \end{column}

  \begin{column}{.6\textwidth}

    \setlength{\parskip}{.5em}

    { \centering

    \input{cc-by-ARTIFACT.pdf_tex}

    \copyright~2022  Red Hat, Inc.

    { \scriptsize
    Except where otherwise noted this work is licensed under
    }
    { \footnotesize
    \textbf{http://creativecommons.org/licenses/by/4.0/}
    }

    }

    \begin{description}
      \item[Slides] \href{https://speakerdeck.com/frasertweedale}{speakerdeck.com/frasertweedale}
      \item[Blog] \href{https://frasertweedale.github.io/blog-redhat/}{frasertweedale.github.io/blog-redhat}
      \item[Email] \texttt{ftweedal@redhat.com}
      \item[Twitter] \href{https://twitter.com/hackuador}{@hackuador}
    \end{description}
  \end{column}

\end{columns}
\end{frame}

\end{document}

\PassOptionsToPackage{unicode=true}{hyperref} % options for packages loaded elsewhere
\PassOptionsToPackage{hyphens}{url}
%
\documentclass[ignorenonframetext,aspectratio=169,12pt]{beamer}
\usepackage{pgfpages}
\setbeamertemplate{caption}[numbered]
\setbeamertemplate{caption label separator}{: }
\setbeamercolor{caption name}{fg=normal text.fg}
\beamertemplatenavigationsymbolsempty
\usepackage{lmodern}
\usepackage{amssymb,amsmath}
\usepackage{ifxetex,ifluatex}
\usepackage{fixltx2e} % provides \textsubscript
\ifnum 0\ifxetex 1\fi\ifluatex 1\fi=0 % if pdftex
  \usepackage[T1]{fontenc}
  \usepackage[utf8]{inputenc}
  \usepackage{textcomp} % provides euro and other symbols
\else % if luatex or xelatex
  \usepackage{unicode-math}
  \defaultfontfeatures{Ligatures=TeX,Scale=MatchLowercase}
\fi
% use upquote if available, for straight quotes in verbatim environments
\IfFileExists{upquote.sty}{\usepackage{upquote}}{}
% use microtype if available
\IfFileExists{microtype.sty}{%
\usepackage[]{microtype}
\UseMicrotypeSet[protrusion]{basicmath} % disable protrusion for tt fonts
}{}
\IfFileExists{parskip.sty}{%
\usepackage{parskip}
}{% else
\setlength{\parindent}{0pt}
\setlength{\parskip}{6pt plus 2pt minus 1pt}
}
\usepackage{hyperref}
\hypersetup{
            pdfborder={0 0 0},
            breaklinks=true}
\urlstyle{same}  % don't use monospace font for urls
\newif\ifbibliography
% Prevent slide breaks in the middle of a paragraph:
\widowpenalties 1 10000
\raggedbottom
\setbeamertemplate{part page}{
\centering
\begin{beamercolorbox}[sep=16pt,center]{part title}
  \usebeamerfont{part title}\insertpart\par
\end{beamercolorbox}
}
\setbeamertemplate{section page}{
\centering
\begin{beamercolorbox}[sep=12pt,center]{part title}
  \usebeamerfont{section title}\insertsection\par
\end{beamercolorbox}
}
\setbeamertemplate{subsection page}{
\centering
\begin{beamercolorbox}[sep=8pt,center]{part title}
  \usebeamerfont{subsection title}\insertsubsection\par
\end{beamercolorbox}
}
\AtBeginPart{
  \frame{\partpage}
}
\AtBeginSection{
  \ifbibliography
  \else
    \frame{\sectionpage}
  \fi
}
\AtBeginSubsection{
  \frame{\subsectionpage}
}
\setlength{\emergencystretch}{3em}  % prevent overfull lines
\providecommand{\tightlist}{%
  \setlength{\itemsep}{0pt}\setlength{\parskip}{0pt}}
\setcounter{secnumdepth}{0}

% set default figure placement to htbp
\makeatletter
\def\fps@figure{htbp}
\makeatother

\DeclareUnicodeCharacter{00A0}{~}
\DeclareUnicodeCharacter{03B4}{$\delta$}
\DeclareUnicodeCharacter{03B5}{$\varepsilon$}
\DeclareUnicodeCharacter{03C9}{$\omega$}
\DeclareUnicodeCharacter{2124}{\mathbb{Z}}
\DeclareUnicodeCharacter{2193}{$\downarrow$}
\DeclareUnicodeCharacter{2208}{$\in$}
\DeclareUnicodeCharacter{2209}{$\notin$}
\DeclareUnicodeCharacter{220B}{$\ni$}
\DeclareUnicodeCharacter{2227}{$\wedge$}
\DeclareUnicodeCharacter{2228}{$\vee$}
\DeclareUnicodeCharacter{2234}{$\therefore$}
\DeclareUnicodeCharacter{2264}{$\leq$}
\DeclareUnicodeCharacter{2265}{$\geq$}
\DeclareUnicodeCharacter{2605}{$\star$}
\DeclareUnicodeCharacter{1D53D}{\mathbb{F}}


\hypersetup{colorlinks,linkcolor=,urlcolor=purple}
\setbeamertemplate{navigation symbols}{}
\usefonttheme[onlymath]{serif}

\setbeamercolor{footnote mark}{fg=gray}
\setbeamerfont{footnote}{size=\tiny}
\usepackage{color}
\usepackage[normalem]{ulem}
\usepackage{listings}
\lstset{
    basicstyle=\ttfamily\normalsize,
    keywordstyle=\color{blue}\bfseries,
    commentstyle=\color[rgb]{0,0.5,0}\bfseries\em,
    stringstyle=\color{red}\bfseries\em,
    escapeinside={(*}{*)}
}

\title{\bf The Haskell Security Advisory Database}
\providecommand{\subtitle}[1]{}
\subtitle{\bf Status and next steps}
\author{{\bf Fraser Tweedale}\\
    \texttt{@hackuador@functional.cafe}}
\date{February 5, 2023}

\begin{document}
\frame{\titlepage}

\begin{frame}{Haskell Security Advisory Database - why}

\begin{itemize}
    \item Security matters
    \item Haskell should have best-in-class security tooling
    \item Increasingly important for industry adoption
    \item Needed for some certifications (e.g. ISO 27001)
\end{itemize}

\end{frame}

\begin{frame}{Haskell Security Advisory Database - history}

\begin{itemize}
    \item August 2022: Haskell Foundation Tech Proposal
        \begin{itemize}
            \item \url{https://github.com/haskellfoundation/tech-proposals/pull/37}
        \end{itemize}
    \item November 2022: Database repository created
        \begin{itemize}
            \item \url{https://github.com/haskell/security-advisories}
        \end{itemize}
    \item February 2023: Haskell {\bf Security Response Team} call for nominations
        \begin{itemize}
            \item \url{https://discourse.haskell.org/t/call-for-volunteers-haskell-security-response-team/5770}
        \end{itemize}
\end{itemize}

\end{frame}

\begin{frame}{Haskell Security Response Team (SRT) - responsibilities}

\begin{itemize}
    \item Triage and assess incoming security reports
    \item Maintain the advisory database
    \item Ensure the advisory database is useful for downstream tools
        \begin{itemize}
            \item e.g. cabal-install, GitHub dependabot
            \item developing these tools is {\bf not an SRT
                responsibility}
        \end{itemize}
    \item Quarterly report on activities of SRT and trends in
        security issues
\end{itemize}

\end{frame}

\begin{frame}{Haskell Security Response Team (SRT) - who}

\begin{itemize}
    \item 5 volunteers who can commit to an initial term of 6 or
        12 months
    \item People with experience in security topics such as:
        \begin{itemize}
            \item secure development, web security
            \item penetration testing, incident response,
                vulnerability research
            \item cryptography, authentication, identity management
            \item governance, risk and compliance (GRC)
            \item et cetera
        \end{itemize}
\end{itemize}

\end{frame}

\begin{frame}{Haskell Security Response Team (SRT) - nominations}

\begin{itemize}
    \item Official {\bf Call for Nominations} will go out in the
        next day or so.
    \item People should self-nominate
    \item Mail \textbf{\texttt{frase+hasksec@frase.id.au}} to nominate
    \item Aiming to announce the initial SRT around the end of February
\end{itemize}

\end{frame}


\begin{frame}[plain]
\begin{columns}

  \begin{column}{.6\textwidth}

    \setlength{\parskip}{.5em}

    { \centering

    \input{cc-by-ARTIFACT.pdf_tex}

    \copyright~2023  Red Hat, Inc.

    { \scriptsize
    Except where otherwise noted this work is licensed under
    }
    { \footnotesize
    \textbf{http://creativecommons.org/licenses/by/4.0/}
    }

    }

    \begin{description}
      \item[Blog] \href{https://frasertweedale.github.io/blog-fp/}{frasertweedale.github.io/blog-fp}
      \item[Fediverse] \href{https://functional.cafe/@hackuador}{@hackuador@functional.cafe}
    \end{description}
  \end{column}

\end{columns}
\end{frame}

\end{document}

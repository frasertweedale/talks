\PassOptionsToPackage{unicode=true}{hyperref} % options for packages loaded elsewhere
\PassOptionsToPackage{hyphens}{url}
%
\documentclass[ignorenonframetext,aspectratio=169,12pt]{beamer}
\usepackage{pgfpages}
\setbeamertemplate{caption}[numbered]
\setbeamertemplate{caption label separator}{: }
\setbeamercolor{caption name}{fg=normal text.fg}
\beamertemplatenavigationsymbolsempty
\usepackage{lmodern}
\usepackage{amssymb,amsmath}
\usepackage{ifxetex,ifluatex}
\usepackage{fixltx2e} % provides \textsubscript
\ifnum 0\ifxetex 1\fi\ifluatex 1\fi=0 % if pdftex
  \usepackage[T1]{fontenc}
  \usepackage[utf8]{inputenc}
  \usepackage{textcomp} % provides euro and other symbols
\else % if luatex or xelatex
  \usepackage{unicode-math}
  \defaultfontfeatures{Ligatures=TeX,Scale=MatchLowercase}
\fi
% use upquote if available, for straight quotes in verbatim environments
\IfFileExists{upquote.sty}{\usepackage{upquote}}{}
% use microtype if available
\IfFileExists{microtype.sty}{%
\usepackage[]{microtype}
\UseMicrotypeSet[protrusion]{basicmath} % disable protrusion for tt fonts
}{}
\IfFileExists{parskip.sty}{%
\usepackage{parskip}
}{% else
\setlength{\parindent}{0pt}
\setlength{\parskip}{6pt plus 2pt minus 1pt}
}
\usepackage{hyperref}
\hypersetup{
            pdfborder={0 0 0},
            breaklinks=true}
\urlstyle{same}  % don't use monospace font for urls
\newif\ifbibliography
% Prevent slide breaks in the middle of a paragraph:
\widowpenalties 1 10000
\raggedbottom
\setbeamertemplate{part page}{
\centering
\begin{beamercolorbox}[sep=16pt,center]{part title}
  \usebeamerfont{part title}\insertpart\par
\end{beamercolorbox}
}
\setbeamertemplate{section page}{
\centering
\begin{beamercolorbox}[sep=12pt,center]{part title}
  \usebeamerfont{section title}\insertsection\par
\end{beamercolorbox}
}
\setbeamertemplate{subsection page}{
\centering
\begin{beamercolorbox}[sep=8pt,center]{part title}
  \usebeamerfont{subsection title}\insertsubsection\par
\end{beamercolorbox}
}
\AtBeginPart{
  \frame{\partpage}
}
\AtBeginSection{
  \ifbibliography
  \else
    \frame{\sectionpage}
  \fi
}
\AtBeginSubsection{
  \frame{\subsectionpage}
}
\setlength{\emergencystretch}{3em}  % prevent overfull lines
\providecommand{\tightlist}{%
  \setlength{\itemsep}{0pt}\setlength{\parskip}{0pt}}
\setcounter{secnumdepth}{0}

% set default figure placement to htbp
\makeatletter
\def\fps@figure{htbp}
\makeatother

\DeclareUnicodeCharacter{00A0}{~}
\DeclareUnicodeCharacter{03B4}{$\delta$}
\DeclareUnicodeCharacter{03B5}{$\varepsilon$}
\DeclareUnicodeCharacter{03C9}{$\omega$}
\DeclareUnicodeCharacter{2124}{\mathbb{Z}}
\DeclareUnicodeCharacter{2193}{$\downarrow$}
\DeclareUnicodeCharacter{2208}{$\in$}
\DeclareUnicodeCharacter{2209}{$\notin$}
\DeclareUnicodeCharacter{220B}{$\ni$}
\DeclareUnicodeCharacter{2227}{$\wedge$}
\DeclareUnicodeCharacter{2228}{$\vee$}
\DeclareUnicodeCharacter{2234}{$\therefore$}
\DeclareUnicodeCharacter{2264}{$\leq$}
\DeclareUnicodeCharacter{2265}{$\geq$}
\DeclareUnicodeCharacter{2605}{$\star$}
\DeclareUnicodeCharacter{1D53D}{\mathbb{F}}


\hypersetup{colorlinks,linkcolor=,urlcolor=purple}
\setbeamertemplate{navigation symbols}{}
\usefonttheme[onlymath]{serif}

\setbeamercolor{footnote mark}{fg=gray}
\setbeamerfont{footnote}{size=\tiny}
\usepackage{color}
\usepackage[normalem]{ulem}
\usepackage{listings}
\lstset{
    basicstyle=\ttfamily\normalsize,
    keywordstyle=\color{blue}\bfseries,
    commentstyle=\color[rgb]{0,0.5,0}\bfseries\em,
    stringstyle=\color{red}\bfseries\em,
    escapeinside={(*}{*)}
}

\title{\bf send in the {\tt chown(2)}s}
\providecommand{\subtitle}[1]{}
\subtitle{\bf systemd containers in user namespaces}
\author{{\bf Fraser Tweedale}\\
    \texttt{@hackuador@functional.cafe}\\
    \bigskip
    \medskip
    \def\svgwidth{4cm}
    \input{Logo-RedHat-A-Color-RGB-ARTIFACT.pdf_tex}}
\date{February 4, 2023}

\begin{document}
\frame{\titlepage}

\begin{frame}{Agenda}
\protect\hypertarget{agenda}{}

\begin{itemize}
    \item Containers and container standards
    \item Kubernetes and OpenShift
    \item User namespaces and cgroups
    \item systemd-based workloads on Kubernetes/OpenShift
\end{itemize}

\end{frame}


\begin{frame}{What is a container?}
\protect\hypertarget{container-definition}{}

\begin{itemize}
    \item An process isolation and confinement {\em abstraction}
    \item Most commonly: OS-level virtualisation (shared kernel)
        \begin{itemize}
        \item e.g. FreeBSD jails, Solaris zones
        \end{itemize}
    \item Container \emph{\textbf{image}} defines filesystem contents
\end{itemize}

\end{frame}


\begin{frame}{Containers on linux}
\protect\hypertarget{container-linux}{}

Some combination of the following security mechanisms:

\begin{itemize}
  \item \textbf{\texttt{namespaces(7)}}: pid, mount, network, cgroup, \ldots{}
    % \begin{itemize} \item  What can I see?   \end{itemize}
  \item restricted \textbf{\texttt{capabilities(7)}} and/or \textbf{\texttt{seccomp(2)}} profile
    % \begin{itemize} \item  What can I do (syscalls)?   \end{itemize}
  \item {\bf SELinux} or AppArmor MAC policy
    % \begin{itemize} \item  What can I access (files)?   \end{itemize}
  \item \textbf{\texttt{cgroups(7)}} for resource limits
    % \begin{itemize} \item  What resources can I consume? \end{itemize}
\end{itemize}

\emph{Not necessarily all of these at the same time\ldots{}}

\end{frame}

\begin{frame}{Container standards}
\protect\hypertarget{container-standards}{}
\begin{itemize}
    \item {\em Open Container Initiative (OCI)\footnote{\url{https://opencontainers.org}}}
    \item {\bf Runtime Specification}\footnote{\url{https://github.com/opencontainers/runtime-spec}} - low level runtime interface
    \begin{itemize}
        \item Linux, Solaris, Windows, VMs, \ldots{}
        \item Implementations\footnote{\url{https://github.com/opencontainers/runtime-spec/blob/main/implementations.md}}:
          {\bf runc}\footnote{\url{https://github.com/opencontainers/runc}} (reference implementation),
          crun\footnote{\url{https://github.com/containers/crun}}, \\
          Kata Containers\footnote{\url{https://katacontainers.io/}}
    \end{itemize}
\end{itemize}
\end{frame}

\begin{frame}{OCI Runtime Specification}
\protect\hypertarget{oci-runtime-spec}{}
\begin{itemize}
\item JSON configuration (example\footnote{\url{https://github.com/opencontainers/runtime-spec/blob/main/config.md\#configuration-schema-example}})
\item mounts, process and environment, lifecycle hooks, \ldots{}
\item Linux-specific: capabilities, namespaces, cgroup, sysctls,
      {\tt seccomp} profile
\end{itemize}
\end{frame}

\begin{frame}[fragile]{OCI Runtime Specification}
\protect\hypertarget{kubernetes-pod-spec-example}{}
\begin{lstlisting}[basicstyle=\ttfamily\footnotesize]
{
  "process": {
    (*\colorbox{yellow}{"user"}*): { "uid": 0, "gid": 0 },
    "args": [ "/sbin/init" ],
    "env": [ ... ],
    ...
  },
  "root": {"path": "/home/ftweedal/scratch/fs", "readonly": false},
  "hostname": "runc",
  "mounts": [ ... ],
  "linux": {
    (*\colorbox{yellow}{"namespaces"}*): [
      { "type": "pid" }, { "type": "ipc" }, { "type": "uts" },
      { "type": "mount" }, { "type": "cgroup" }
    ],
    "cgroupsPath": "user.slice:runc:sandbox"
  }
}
\end{lstlisting}
\end{frame}



\begin{frame}[plain]
\centering
\huge Kubernetes and OpenShift
\end{frame}


\begin{frame}{Kubernetes - container orchestration}
\protect\hypertarget{kubernetes-intro}{}

\begin{itemize}
\item A container orchestration system
\item Declarative configuration of container-based applications
\item Integration with many cloud providers
\item \url{https://kubernetes.io/}
\item \url{https://github.com/kubernetes/}
\end{itemize}

\end{frame}


\begin{frame}{Kubernetes - terminology}
\protect\hypertarget{kubernetes-terminology}{}

\begin{itemize}
\item {\bf Container}: isolated/confined process [tree]
\item {\bf Pod}: group (1+) of related Containers (e.g. HTTP app + database)
\item {\bf Namespace}: object and auth[nz] scope, such as for a team/project
\item {\bf Node}: a machine in the cluster; where Pods are executed
\end{itemize}

\end{frame}

\begin{frame}{Kubernetes - more terminology}
\protect\hypertarget{kubernetes-terminology-2}{}
\begin{itemize}

\item {\bf
  Kubelet}\footnote{\url{https://kubernetes.io/docs/reference/command-line-tools-reference/kubelet/}}:
  agent that executes Pods on Nodes

\item {\bf Sandbox}: isolation/confinement mechanism(s); one per Pod
\item {\bf Container Runtime Interface
  (CRI)}\footnote{\url{https://kubernetes.io/docs/concepts/architecture/cri/}}:
  interface used by Kubelet to create/start/stop/destroy Sandboxes and
  Containers
  \begin{itemize}
  \item CRI-O\footnote{\url{https://cri-o.io/}}
  \item containerd\footnote{\url{https://containerd.io/}}
  \end{itemize}
\end{itemize}
\end{frame}

\begin{frame}{Kubernetes - Container Runtime Interface}
\begin{center}
\def\svgwidth{\textwidth}
\input{figure-cri-ARTIFACT.pdf_tex}
\end{center}
\end{frame}

\begin{frame}{Kubernetes - Container Runtime Interface - CRI-O}
\begin{center}
\def\svgwidth{\textwidth}
\input{figure-cri-crio-oci-ARTIFACT.pdf_tex}
\end{center}
\end{frame}

\begin{frame}{Kubernetes - Container Runtime Interface - CRI-O + runc}
\begin{center}
\def\svgwidth{\textwidth}
\input{figure-cri-crio-runc-ARTIFACT.pdf_tex}
\end{center}
\end{frame}

\begin{frame}[fragile]{Kubernetes - Pod definition}
\protect\hypertarget{kubernetes-pod-spec}{}
\begin{lstlisting}
apiVersion: v1
(*\colorbox{yellow}{kind: Pod}*)
metadata:
  name: fedora
  labels:
    app: fedora
spec:
  containers:
  - name: fedora
    (*\colorbox{yellow}{image}*): registry.fedoraproject.org/fedora:35-x86_64
    (*\colorbox{yellow}{command}*): ["sleep", "3600"]
    (*\colorbox{yellow}{env}*):
    - name: DEBUG
      value: "1"
\end{lstlisting}
\end{frame}

\begin{frame}{OpenShift\footnote{\url{https://openshift.com/}}}
\protect\hypertarget{openshift-intro}{}
\begin{itemize}
\item a.k.a. {\em OpenShift Container Platform (OCP)}
\item An {\em enterprise-ready Kubernetes container platform}
\item Commercially supported by Red Hat
\item Community ``upstream'' distribution: OKD\footnote{\url{https://www.okd.io/}}
\item Uses {\bf CRI-O} and {\bf runc}
\item Latest stable release: 4.12
\end{itemize}
\end{frame}

\begin{frame}{OpenShift runtime environment (default)}
\protect\hypertarget{openshift-runtime-today}{}
\begin{itemize}
\item Confinement: SELinux, namespaces (cgroup, pid, mount, uts, network)
\item Each {\em namespace} gets assigned a unique uid range
\item Containers run as a uid from that range
    \begin{itemize}
      \item Circumvent via {\tt RunAsUser} and {\em Security Context
        Constraints (SCCs)}
      \item \dots{}which is a {\bf bad idea}
    \end{itemize}
\end{itemize}
\end{frame}


\begin{frame}[plain]
\centering
\huge User namespaces
\end{frame}

\begin{frame}{User namespaces - why}
\protect\hypertarget{userns-why}{}
\begin{itemize}
\item Increase workload isolation and confinement
\item Run applications that require/assume specific uid(s)
\end{itemize}
\end{frame}

\begin{frame}{User namespaces - why}
\protect\hypertarget{userns-why-cves}{}
  \centering
  \small
  \def\arraystretch{1.5}
  \begin{tabular}{|l|p{0.5\textwidth}|c|}
      \centering {\bf CVE} & \centering {\bf description} & {\bf CVSS}\footnote{{\bf NIST} or {\bf NIST / CNA Kubernetes}} \\
    \hline
    CVE-2019-5736    & host runc binary overwritten from container & 8.6 \\
    CVE-2021-25741   & host fs access via symlink exchange attack & 8.1 / 8.6 \\
    CVE-2021-30465   & host fs access via symlink exchange attack & 8.5 \\
    CVE-2017-1002101 & host fs access via subpath volume mount & 9.6 / 8.8 \\
    CVE-2016-8867    & privesc due to excessive ambient capabilities & 7.5 \\
    CVE-2018-15664   & host fs access via subpath volume mount & 7.5 \\
  \end{tabular}
\end{frame}

\begin{frame}{User namespaces - what}
\protect\hypertarget{userns-what}{}
\begin{center}
\def\svgwidth{\textwidth}
\input{figure-user-namespaces-ARTIFACT.pdf_tex}
\end{center}
\end{frame}

\begin{frame}[fragile]{User namespaces - how (Linux)}
\protect\hypertarget{userns-how-linux}{}
\begin{itemize}
    \item {\tt user\_namespaces(7)}
    \item {\tt unshare(2)}
    \item {\tt unshare(1)}
\begin{lstlisting}
% id -u
1000
% unshare --user --map-root-user id -u
0
\end{lstlisting}
\end{itemize}
\end{frame}

\begin{frame}[fragile]{User namespaces - how (OCI runtime)}
\protect\hypertarget{userns-how-oci}{}
\begin{lstlisting}[basicstyle=\ttfamily\small]
{
  ...
  "linux": {
    ...
    "namespaces": [
      { "type": "user" },
      ...
    ],
    "uidMappings": [
      { "containerID": 0, "hostID": 1000000, "size": 65536 }
    ],
    "gidMappings": [
      { "containerID": 0, "hostID": 1000000, "size": 65536 }
    ]
  }
}
\end{lstlisting}
\end{frame}

\begin{frame}{User namespaces - how (OpenShift)}
\protect\hypertarget{userns-how-openshift}{}
\begin{itemize}
  \item Implemented in CRI-O
    \begin{itemize}
      \item OpenShift 4.7 (with non-default cluster config)
      \item OpenShift 4.10 (out of the box)
    \end{itemize}
  \item Opt-in via Pod {\em annotations}
  \item Requires \textbf{\texttt{anyuid}} SCC (or equivalent) for {\bf admission}
  \item Some workloads may still require non-default cluster
    configuration
\end{itemize}
\end{frame}

\begin{frame}[fragile]{User namespaces - how (OpenShift)}
\protect\hypertarget{userns-how-openshift-2}{}
\begin{lstlisting}
apiVersion: v1
kind: Pod
metadata:
  name: nginx
  labels:
    app: nginx
  annotations:
    (*\colorbox{yellow}{io.openshift.builder: "true"}*)
    (*\colorbox{yellow}{io.kubernetes.cri-o.userns-mode: "auto:size=65536"}*)
spec:
  containers:
  - name: nginx
    image: quay.io/ftweedal/test-nginx:latest
    tty: true
\end{lstlisting}
\end{frame}

\begin{frame}{User namespaces - how (Kubernetes)}
\protect\hypertarget{userns-how-kubernetes}{}
\begin{itemize}
\item KEP\footnote{Kubernetes Enhancement Proposal}-127\footnote{\url{https://github.com/kubernetes/enhancements/pull/3065}}
\item Initial support delivered in k8s v1.25 (OpenShift 4.12)
    \begin{itemize}
        \item Alpha; feature gate: \textbf{\texttt{UserNamespacesStatelessPodsSupport}}
        \item Supported volume types: emptyDir, configmap, secret
        \item Post must opt in by setting \textbf{\texttt{spec.hostUsers: false}}
        \item Fixed mapping size (65536); unique to Pod
    \end{itemize}
\item Support for more volume types deferred to later phase
\end{itemize}
\end{frame}

\begin{frame}{User namespaces - how (Kubernetes) - challenges}
\protect\hypertarget{userns-how-kubernetes-3}{}
\begin{itemize}
\item shared / persistent volumes require \textbf{\em ID-mapped mounts}
\item {\bf simple heuristics} for ID range assignment $\to$ lower number
  of pods with unique user namespaces
\item other mount point and file ownership issues (e.g. cgroupfs)
\end{itemize}
\end{frame}


\begin{frame}{cgroups}
\protect\hypertarget{openshift-cgroup}{}
\begin{itemize}

\item OpenShift creates a {\bf unique cgroup}\footnote{\tt cgroups(7)} for each container

\item {\bf cgroup namespace}\footnote{\tt cgroup\_namespaces(7)} makes it the ``root''
      namespace inside the container

\item cgroupfs mounts it at {\tt /sys/fs/cgroup}

\item {\tt systemd} needs write access\ldots{} but doesn't have it

\end{itemize}
\end{frame}

\begin{frame}{cgroups - send in the \texttt{chown(2)}s}
\protect\hypertarget{openshift-cgroup-ownership}{}
\begin{itemize}

\item Solution: modify runtime to {\tt chown} the cgroup to the container
      process UID

\item But first: extend OCI Runtime Spec with semantics for
      cgroup ownership\footnote{\url{https://github.com/opencontainers/runtime-spec/blob/main/config-linux.md\#cgroup-ownership}}

\item runc pull request\footnote{\url{https://github.com/opencontainers/runc/pull/3057}}
  \begin{itemize}
  \item Merged; released in OpenShift 4.11
  \end{itemize}

\end{itemize}
\end{frame}

\begin{frame}{cgroups - OCI runtime semantics}
\protect\hypertarget{openshift-cgroup-ownership-semantics}{}
{\tt chown} container's cgroup to host UID matching the process UID in
container's user namespace, if and only if\ldots{}
\begin{itemize}
\item {\bf cgroups v2} in use; and
\item container has its own cgroup namespace; and
\item cgroupfs is mounted read/write
\end{itemize}
\end{frame}

\begin{frame}{cgroups - OCI runtime semantics}
\protect\hypertarget{openshift-cgroup-ownership-semantics-files}{}
Only the {\bf cgroup directory itself}, and the files mentioned in {\tt
/sys/kernel/cgroup/delegate}, should be {\tt chown}'d:

\begin{itemize}
\item {\tt cgroup.procs}
\item {\tt cgroup.threads}
\item {\tt cgroup.subtree\_control}
\item {\tt memory.oom.group}\footnote{depends on kernel version}
\item {\tt memory.reclaim}\footnotemark[\value{footnote}]
\end{itemize}

\end{frame}


\begin{frame}{cgroups - OpenShift}
\protect\hypertarget{openshift-cgroups-v2}{}
\begin{itemize}
\item cgroups v2 not the default (yet)
\item but it works and is supported
\item annotation required to activate the ownership semantics: \\
    \textbf{\texttt{io.kubernetes.cri-o.cgroup2-mount-hierarchy-rw: "true"}}
\end{itemize}
\end{frame}


\begin{frame}[plain]
\centering
\huge Demo
\end{frame}


\begin{frame}{Links / resources}
\protect\hypertarget{links}{}

\begin{itemize}
\item My blog: \url{https://frasertweedale.github.io/blog-redhat/tags/containers.html}
\item Demo: \url{https://www.youtube.com/watch?v=OGAVvIJwmd0}
\item KEP-127: \url{https://github.com/kubernetes/enhancements/tree/master/keps/sig-node/127-user-namespaces}
\item OCI Runtime Specification: \url{https://github.com/opencontainers/runtime-spec}
\end{itemize}

\end{frame}


\begin{frame}[plain]
\begin{columns}

  \begin{column}{.6\textwidth}

    \setlength{\parskip}{.5em}

    { \centering

    \input{cc-by-ARTIFACT.pdf_tex}

    \copyright~2023  Red Hat, Inc.

    { \scriptsize
    Except where otherwise noted this work is licensed under
    }
    { \footnotesize
    \textbf{http://creativecommons.org/licenses/by/4.0/}
    }

    }

    \begin{description}
      \item[Slides] \href{https://speakerdeck.com/frasertweedale}{speakerdeck.com/frasertweedale}
      \item[Blog] \href{https://frasertweedale.github.io/blog-redhat/}{frasertweedale.github.io/blog-redhat}
      \item[Email] \texttt{ftweedal@redhat.com}
      \item[Fediverse] \href{https://functional.cafe/@hackuador}{@hackuador@functional.cafe}
    \end{description}
  \end{column}

\end{columns}
\end{frame}

\end{document}

\PassOptionsToPackage{unicode=true}{hyperref} % options for packages loaded elsewhere
\PassOptionsToPackage{hyphens}{url}
%
\documentclass[ignorenonframetext,aspectratio=169,12pt]{beamer}
\usepackage{pgfpages}
\setbeamertemplate{caption}[numbered]
\setbeamertemplate{caption label separator}{: }
\setbeamercolor{caption name}{fg=normal text.fg}
\beamertemplatenavigationsymbolsempty
\usepackage{lmodern}
\usepackage{amssymb,amsmath}
\usepackage{ifxetex,ifluatex}
\usepackage{fixltx2e} % provides \textsubscript
\ifnum 0\ifxetex 1\fi\ifluatex 1\fi=0 % if pdftex
  \usepackage[T1]{fontenc}
  \usepackage[utf8]{inputenc}
  \usepackage{textcomp} % provides euro and other symbols
\else % if luatex or xelatex
  \usepackage{unicode-math}
  \defaultfontfeatures{Ligatures=TeX,Scale=MatchLowercase}
\fi
% use upquote if available, for straight quotes in verbatim environments
\IfFileExists{upquote.sty}{\usepackage{upquote}}{}
% use microtype if available
\IfFileExists{microtype.sty}{%
\usepackage[]{microtype}
\UseMicrotypeSet[protrusion]{basicmath} % disable protrusion for tt fonts
}{}
\IfFileExists{parskip.sty}{%
\usepackage{parskip}
}{% else
\setlength{\parindent}{0pt}
\setlength{\parskip}{6pt plus 2pt minus 1pt}
}
\usepackage{hyperref}
\hypersetup{
            pdfborder={0 0 0},
            breaklinks=true}
\urlstyle{same}  % don't use monospace font for urls
\newif\ifbibliography
% Prevent slide breaks in the middle of a paragraph:
\widowpenalties 1 10000
\raggedbottom
\setbeamertemplate{part page}{
\centering
\begin{beamercolorbox}[sep=16pt,center]{part title}
  \usebeamerfont{part title}\insertpart\par
\end{beamercolorbox}
}
\setbeamertemplate{section page}{
\centering
\begin{beamercolorbox}[sep=12pt,center]{part title}
  \usebeamerfont{section title}\insertsection\par
\end{beamercolorbox}
}
\setbeamertemplate{subsection page}{
\centering
\begin{beamercolorbox}[sep=8pt,center]{part title}
  \usebeamerfont{subsection title}\insertsubsection\par
\end{beamercolorbox}
}
\AtBeginPart{
  \frame{\partpage}
}
\AtBeginSection{
  \ifbibliography
  \else
    \frame{\sectionpage}
  \fi
}
\AtBeginSubsection{
  \frame{\subsectionpage}
}
\setlength{\emergencystretch}{3em}  % prevent overfull lines
\providecommand{\tightlist}{%
  \setlength{\itemsep}{0pt}\setlength{\parskip}{0pt}}
\setcounter{secnumdepth}{0}

% set default figure placement to htbp
\makeatletter
\def\fps@figure{htbp}
\makeatother

\DeclareUnicodeCharacter{00A0}{~}
\DeclareUnicodeCharacter{03B4}{$\delta$}
\DeclareUnicodeCharacter{03B5}{$\varepsilon$}
\DeclareUnicodeCharacter{03C9}{$\omega$}
\DeclareUnicodeCharacter{2124}{\mathbb{Z}}
\DeclareUnicodeCharacter{2193}{$\downarrow$}
\DeclareUnicodeCharacter{2208}{$\in$}
\DeclareUnicodeCharacter{2209}{$\notin$}
\DeclareUnicodeCharacter{220B}{$\ni$}
\DeclareUnicodeCharacter{2227}{$\wedge$}
\DeclareUnicodeCharacter{2228}{$\vee$}
\DeclareUnicodeCharacter{2234}{$\therefore$}
\DeclareUnicodeCharacter{2264}{$\leq$}
\DeclareUnicodeCharacter{2265}{$\geq$}
\DeclareUnicodeCharacter{2605}{$\star$}
\DeclareUnicodeCharacter{1D53D}{\mathbb{F}}


\hypersetup{colorlinks,linkcolor=,urlcolor=purple}
\setbeamertemplate{navigation symbols}{}
\usefonttheme[onlymath]{serif}

\setbeamercolor{footnote mark}{fg=gray}
\setbeamerfont{footnote}{size=\tiny}
\usepackage{color}
\usepackage[normalem]{ulem}
\usepackage{listings}
\lstset{
    basicstyle=\ttfamily\normalsize,
    keywordstyle=\color{blue}\bfseries,
    commentstyle=\color[rgb]{0,0.5,0}\bfseries\em,
    stringstyle=\color{red}\bfseries\em,
    escapeinside={(*}{*)}
}

\title{\bf Advent of Code 2023}
\providecommand{\subtitle}[1]{}
\subtitle{A Haskeller's reflections}
\author{{\bf Fraser Tweedale}\\
    \texttt{@hackuador@functional.cafe}}
\date{2024-02-13}

\begin{document}
\frame{\titlepage}

\begin{frame}{Advent of Code}

    \large
    \bigskip
    \bigskip

    \begin{quote}
    Advent of Code is an Advent calendar of small programming
    puzzles for a variety of skill sets and skill levels that can be
    solved in any programming language you like. People use them as
    interview prep, company training, university coursework,
    practice problems, a speed contest, or to challenge each other.
    \end{quote}

    \begin{flushright}
    ---\url{https://adventofcode.com/2023/about}
    \end{flushright}

\end{frame}

\begin{frame}{Advent of Code}

    \begin{itemize}
        \item Founded in 2015 by Eric Wastl
        \item Two-part puzzle released each day during December
        \item Unique puzzle input (and solution) for each user
        \item Global and private leaderboards (fastest correct submissions)
    \end{itemize}

\end{frame}

\begin{frame}{Advent of Code - puzzle structure}

    \begin{itemize}
        \item Part 1:
            \begin{itemize}
                \item problem description + sample input and result
                \item unique puzzle input ($\gg$ sample)
            \end{itemize}
        \item Part 2: 
            \begin{itemize}
                \item revealed when correct Part 1 answer submitted
                \item {\bf same input}, reinterpreted in some way
                \item sample input and result
            \end{itemize}
    \end{itemize}

\end{frame}

\begin{frame}{Advent of Code - part 1 vs part 2}

    \begin{itemize}
        \item generalise the part 1 solution
        \item naive/brute force $\to$ efficient algorithm
        \item sometimes a complete change of approach
    \end{itemize}

\end{frame}


\begin{frame}{Advent of Code - Day 1, 2023}

    \centering
    \LARGE
    \url{https://adventofcode.com/2023/day/1}

\end{frame}


\begin{frame}[fragile]{Advent of Code - problem difficulty}

Problems tend toward higher complexity/difficulty:

\begin{verbatim}
   42 src/Y2023/D01.hs      107 src/Y2023/D12.hs
   78 src/Y2023/D02.hs       68 src/Y2023/D13.hs
  104 src/Y2023/D03.hs       71 src/Y2023/D14.hs
   55 src/Y2023/D04.hs       88 src/Y2023/D15.hs
  130 src/Y2023/D05.hs       96 src/Y2023/D16.hs
   46 src/Y2023/D06.hs      126 src/Y2023/D17.hs
  118 src/Y2023/D07.hs      161 src/Y2023/D18.hs
  104 src/Y2023/D08.hs      183 src/Y2023/D19.hs
   47 src/Y2023/D09.hs      204 src/Y2023/D20.hs
  164 src/Y2023/D10.hs      192 src/Y2023/D21.hs
   68 src/Y2023/D11.hs
\end{verbatim}

\end{frame}


\begin{frame}{Advent of Code - my results}

    \begin{itemize}
        \item Most problems took < 2h, some much longer
        \item I stopped after Day 21 (got too busy)
        \item Didn't make the daily "Top 100" leaderboard (sometimes close)
    \end{itemize}

\end{frame}



\begin{frame}{Advent of Code - tips}

    \begin{itemize}
        \item {\bf No invalid inputs}---you don't need to handle them.
        \item The sample input is given for a reason.  Test your code!
        \item Write lots of {\bf comments} (as you go) to explain
            your solution.
        \item Don't be afraid to use {\bf pen and paper}!
    \end{itemize}


\end{frame}


\begin{frame}{Advent of Code - gotchas}

The problem description does not always contain everything you need
to solve the problem.

{\bf Occasionally it depends on properties of the input, which you must
    discover yourself.}

e.g. \url{https://adventofcode.com/2023/day/21}

\end{frame}


\begin{frame}{Advent of Code - common topics}

    Most common:

    \begin{itemize}
        \item parsing
        \item list processing
        \item graphs (shortest path, cycles)
        \item matrices (spacial problems)
        \item state machines
    \end{itemize}

    Occasional:

    \begin{itemize}
        \item combinatorics
        \item dynamic programming
    \end{itemize}

\end{frame}


\begin{frame}{Advent of Code - data structures and algorithms}

    Frequently needed:

    \begin{itemize}
        \item {\bf list} / non-empty list
            \begin{itemize}
                \item {\bf matrix} = list-of-list
            \end{itemize}
        \item {\bf map/dict} and set
        \item {\bf graph}
            \begin{itemize}
                \item directed and undirected
                \item adjacency-list and matrix representations
                \item DFS, BFS, shortest path, cycle detection
            \end{itemize}
    \end{itemize}

    Not used (yet\ldots{}):

    \begin{itemize}
        \item tree, heap, stack, zipper
    \end{itemize}

\end{frame}

\section{Using Haskell}

\begin{frame}{Advent of Code 2023 - self-imposed constraints}

Limited to the Haskell {\em base} library:

\begin{itemize}
    \item had to implement {\tt Map} type myself
        (\href{https://en.wikipedia.org/wiki/AVL_tree}{AVL tree})
    \item had to implement parsing library myself
    \item using mature libraries (e.g. {\em vector}, {\em
        containers}) would reduce development time; solutions
        somewhat faster
\end{itemize}

It was worthwhile to review some fundamentals.

I would use off-the-shelf libraries next time.

\end{frame}


\begin{frame}{Advent of Code - where Haskell shines}

    \begin{itemize}
        \item parsing
        \item list processing (incl. strings)
        \item laziness
        \item state machines
        \item problem-specific data domains (e.g. playing cards)
    \end{itemize}

\end{frame}


\begin{frame}{Advent of Code - where Haskell does not shine}

    \begin{itemize}
        \item dynamic programming (memoisation)
        \item space leaks ({\em deepseq} can help)
        \item debugging
    \end{itemize}

\end{frame}

\section{Let's look at some problems}

\begin{frame}{Final remarks}

    \begin{itemize}
        \item Advent of Code is fun
        \item Do it in whatever way you want
            \begin{itemize}
                \item learn a new language
                \item practice fundamentals
                \item space it out
                \item challenge your friends/workmates
                \item \ldots{}or solve it together
            \end{itemize}
        \item \href{https://www.reddit.com/r/adventofcode/}{\bf r/adventofcode} for tips/ideas/help
        \item Haskell is a great language to do Advent of Code
        \item My code is available:
            \href{https://github.com/frasertweedale/advent}{\bf github.com/frasertweedale/advent}
    \end{itemize}

\end{frame}


\end{frame}

% END MATTER

\begin{frame}[plain]
\begin{columns}

  \begin{column}{.6\textwidth}

    \setlength{\parskip}{.5em}

    { \centering

    \input{cc-by-ARTIFACT.pdf_tex}

    \copyright~2024  Fraser Tweedale

    { \scriptsize
    Except where otherwise noted this work is licensed under
    }
    { \footnotesize
    \textbf{http://creativecommons.org/licenses/by/4.0/}
    }

    }

    \begin{description}
      \item[Code] \url{https://github.com/frasertweedale}
      \item[Blog] \href{https://frasertweedale.github.io/blog-fp/}{frasertweedale.github.io/blog-fp}
      \item[Fediverse] \href{https://functional.cafe/@hackuador}{@hackuador@functional.cafe}
    \end{description}
  \end{column}

\end{columns}
\end{frame}

\end{document}

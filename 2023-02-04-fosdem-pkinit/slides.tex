\PassOptionsToPackage{unicode=true}{hyperref} % options for packages loaded elsewhere
\PassOptionsToPackage{hyphens}{url}
%
\documentclass[ignorenonframetext,aspectratio=169,12pt]{beamer}
\usepackage{pgfpages}
\setbeamertemplate{caption}[numbered]
\setbeamertemplate{caption label separator}{: }
\setbeamercolor{caption name}{fg=normal text.fg}
\beamertemplatenavigationsymbolsempty
\usepackage{lmodern}
\usepackage{amssymb,amsmath}
\usepackage{ifxetex,ifluatex}
\usepackage{fixltx2e} % provides \textsubscript
\ifnum 0\ifxetex 1\fi\ifluatex 1\fi=0 % if pdftex
  \usepackage[T1]{fontenc}
  \usepackage[utf8]{inputenc}
  \usepackage{textcomp} % provides euro and other symbols
\else % if luatex or xelatex
  \usepackage{unicode-math}
  \defaultfontfeatures{Ligatures=TeX,Scale=MatchLowercase}
\fi
% use upquote if available, for straight quotes in verbatim environments
\IfFileExists{upquote.sty}{\usepackage{upquote}}{}
% use microtype if available
\IfFileExists{microtype.sty}{%
\usepackage[]{microtype}
\UseMicrotypeSet[protrusion]{basicmath} % disable protrusion for tt fonts
}{}
\IfFileExists{parskip.sty}{%
\usepackage{parskip}
}{% else
\setlength{\parindent}{0pt}
\setlength{\parskip}{6pt plus 2pt minus 1pt}
}
\usepackage{hyperref}
\hypersetup{
            pdfborder={0 0 0},
            breaklinks=true}
\urlstyle{same}  % don't use monospace font for urls
\newif\ifbibliography
% Prevent slide breaks in the middle of a paragraph:
\widowpenalties 1 10000
\raggedbottom
\setbeamertemplate{part page}{
\centering
\begin{beamercolorbox}[sep=16pt,center]{part title}
  \usebeamerfont{part title}\insertpart\par
\end{beamercolorbox}
}
\setbeamertemplate{section page}{
\centering
\begin{beamercolorbox}[sep=12pt,center]{part title}
  \usebeamerfont{section title}\insertsection\par
\end{beamercolorbox}
}
\setbeamertemplate{subsection page}{
\centering
\begin{beamercolorbox}[sep=8pt,center]{part title}
  \usebeamerfont{subsection title}\insertsubsection\par
\end{beamercolorbox}
}
\AtBeginPart{
  \frame{\partpage}
}
\AtBeginSection{
  \ifbibliography
  \else
    \frame{\sectionpage}
  \fi
}
\AtBeginSubsection{
  \frame{\subsectionpage}
}
\setlength{\emergencystretch}{3em}  % prevent overfull lines
\providecommand{\tightlist}{%
  \setlength{\itemsep}{0pt}\setlength{\parskip}{0pt}}
\setcounter{secnumdepth}{0}

% set default figure placement to htbp
\makeatletter
\def\fps@figure{htbp}
\makeatother

\DeclareUnicodeCharacter{00A0}{~}
\DeclareUnicodeCharacter{03B4}{$\delta$}
\DeclareUnicodeCharacter{03B5}{$\varepsilon$}
\DeclareUnicodeCharacter{03C9}{$\omega$}
\DeclareUnicodeCharacter{2124}{\mathbb{Z}}
\DeclareUnicodeCharacter{2193}{$\downarrow$}
\DeclareUnicodeCharacter{2208}{$\in$}
\DeclareUnicodeCharacter{2209}{$\notin$}
\DeclareUnicodeCharacter{220B}{$\ni$}
\DeclareUnicodeCharacter{2227}{$\wedge$}
\DeclareUnicodeCharacter{2228}{$\vee$}
\DeclareUnicodeCharacter{2234}{$\therefore$}
\DeclareUnicodeCharacter{2264}{$\leq$}
\DeclareUnicodeCharacter{2265}{$\geq$}
\DeclareUnicodeCharacter{2605}{$\star$}
\DeclareUnicodeCharacter{1D53D}{\mathbb{F}}


\hypersetup{colorlinks,linkcolor=,urlcolor=purple}
\setbeamertemplate{navigation symbols}{}
\usefonttheme[onlymath]{serif}

\setbeamercolor{footnote mark}{fg=gray}
\setbeamerfont{footnote}{size=\tiny}
\usepackage{color}
\usepackage[normalem]{ulem}
\usepackage{listings}
\lstset{
    basicstyle=\ttfamily\normalsize,
    keywordstyle=\color{blue}\bfseries,
    commentstyle=\color[rgb]{0,0.5,0}\bfseries\em,
    stringstyle=\color{red}\bfseries\em,
    escapeinside={(*}{*)}
}

\title{\bf Kerberos PKINIT}
\providecommand{\subtitle}[1]{}
\subtitle{\bf what, why and how (to break it)}
\author{{\bf Fraser Tweedale}\\
    \texttt{@hackuador@functional.cafe}\\
    \bigskip
    \def\svgwidth{4cm}
    \input{Logo-RedHat-A-Color-RGB-ARTIFACT.pdf_tex}}
\date{February 4, 2023}

\begin{document}
\frame{\titlepage}

\begin{frame}{Agenda}
\protect\hypertarget{agenda}{}

\begin{itemize}
    \item Kerberos overview
    \item Kerberos PKINIT: overview and demo
    \item PKINIT security considerations
    \item Attack demo: CVE-2022-4254
\end{itemize}

\end{frame}


\begin{frame}{Kerberos - overview}
\protect\hypertarget{kerberos-overview}{}
\begin{itemize}
    \item Authentication protocol based on symmetric cryptography
    \item {\em Single sign-on}: authenticate once, access many services
    \item Started at MIT (1988), v5 (1993),
        RFC 4120\footnote{\url{https://www.rfc-editor.org/rfc/rfc4120}} (2005)
    \item Major implementations:
        MIT Kerberos\footnote{\url{https://web.mit.edu/kerberos/}},
        Microsoft Active Directory,
        Heimdal\footnote{\url{https://github.com/heimdal/heimdal}},
        FreeIPA\footnote{\url{https://www.freeipa.org}} / Identity Management in RHEL
\end{itemize}
\end{frame}


\begin{frame}{Kerberos - protocol}
\protect\hypertarget{kerberos-protocol}{}
\begin{itemize}
    \item Parties: client, {\em Key Distribution Centre (KDC)}, service
        \begin{itemize}
            \item KDC = {\em Authentication Service (AS)} +
                {\em Ticket Granting Service (TGS)}
        \end{itemize}
    \item Users, services, KDC are represented as \textbf{\em principals} in a \textbf{\em realm}
    \item Each principle has a long-term \textbf{\em secret key}, shared with the KDC
        \begin{itemize}
            \item users derive it from a password
            \item hosts/services store it in a file ("keytab")
        \end{itemize}
    \item Authentication tokens are called \textbf{\em tickets}
\end{itemize}
\end{frame}


\begin{frame}{Kerberos - protocol}
\begin{center}
\def\svgwidth{\textwidth}
\input{kerb-parties-ARTIFACT.pdf_tex}
\end{center}
\end{frame}

\begin{frame}{Kerberos - protocol}
\begin{center}
\def\svgwidth{\textwidth}
\input{kerb-as-req-ARTIFACT.pdf_tex}
\end{center}
\end{frame}

\begin{frame}{Kerberos - protocol}
\begin{center}
\def\svgwidth{\textwidth}
\input{kerb-as-rep-ARTIFACT.pdf_tex}
\end{center}
\end{frame}

\begin{frame}{Kerberos - protocol}
\begin{center}
\def\svgwidth{\textwidth}
\input{kerb-as-exchange-post-ARTIFACT.pdf_tex}
\end{center}
\end{frame}

\begin{frame}{Kerberos - protocol}
\begin{center}
\def\svgwidth{\textwidth}
\input{kerb-tgs-req-ARTIFACT.pdf_tex}
\end{center}
\end{frame}

\begin{frame}{Kerberos - protocol}
\begin{center}
\def\svgwidth{\textwidth}
\input{kerb-tgs-rep-ARTIFACT.pdf_tex}
\end{center}
\end{frame}

\begin{frame}{Kerberos - protocol}
\begin{center}
\def\svgwidth{\textwidth}
\input{kerb-tgs-rep-post-ARTIFACT.pdf_tex}
\end{center}
\end{frame}

\begin{frame}{Kerberos - protocol}
\begin{center}
\def\svgwidth{\textwidth}
\input{kerb-ap-req-ARTIFACT.pdf_tex}
\end{center}
\end{frame}

\begin{frame}{Kerberos - protocol}
\begin{center}
\def\svgwidth{\textwidth}
\input{kerb-ap-req-post-ARTIFACT.pdf_tex}
\end{center}
\end{frame}

\begin{frame}{Kerberos - extensions and integrations}
\protect\hypertarget{kerberos-integrations}{}

\begin{itemize}
    \item Pre-authentication framework\footnote{RFC 6113 - \url{https://www.rfc-editor.org/rfc/rfc6113}}
        \begin{itemize}
            \item integrate additional authentication mechanisms
                e.g. {\bf OTP}\footnote{\url{https://web.mit.edu/kerberos/krb5-devel/doc/admin/otp.html}}
        \end{itemize}
    \item {\bf GSSAPI} mechanism\footnote{RFC 4121 - \url{https://www.rfc-editor.org/rfc/rfc4121}}
    \item {\bf SASL} mechanism\footnote{RFC 4752 - \url{https://www.rfc-editor.org/rfc/rfc4752}}
    \item {\bf HTTP} authentication (SPNEGO)\footnote{RFC 4178 - \url{https://www.rfc-editor.org/rfc/rfc4178}}
    \item {\bf Authentication Indicator}\footnote{RFC 8129 - \url{https://www.rfc-editor.org/rfc/rfc8129}}
    \item Cross-realm authentication (trusts)
\end{itemize}

\end{frame}

\begin{frame}{Kerberos - advantages}
\protect\hypertarget{kerberos-advantages}{}

\begin{itemize}
    \item {\bf Single sign-on}: improve efficiency, reduce password fatigue
    \item Client exposes long-term secret only once (until TGT expires)
    \item Resistant to replay attacks
    \item Works well for HTTP and "bare" network protocols
\end{itemize}

\end{frame}

\begin{frame}{Kerberos - problems}
\protect\hypertarget{kerberos-problems}{}

\begin{itemize}
    \item Passwords are not great
    \item Password / keytab rotation is burdensome
\end{itemize}

\end{frame}


\begin{frame}{PKINIT - overview}
\protect\hypertarget{pkinit-overview}{}
\begin{itemize}
    \item \textbf{\em Public Key Cryptography for Initial Authentication in
        Kerberos}\footnote{RFC 4556 - \url{https://www.rfc-editor.org/rfc/rfc4556}}
    \item Client uses {\em asymmetric cryptography} to authenticate to KDC
    \item Client presents
        {\bf X.509}\footnote{RFC 5280 - \url{https://www.rfc-editor.org/rfc/rfc5280}}
        certificate and signs message
    \item KDC verifies certificate, signature and client binding
    \item KDC encrypts response using either:
        \begin{itemize}
            \item Diffie-Hellman (DH) or analogous key agreement algorithm
            \item other public-key encryption algorithm
        \end{itemize}
\end{itemize}
\end{frame}

\begin{frame}{PKINIT - protocol}
\begin{center}
\def\svgwidth{\textwidth}
\input{kerb-pkinit-as-req-ARTIFACT.pdf_tex}
\end{center}
\end{frame}

\begin{frame}{PKINIT - protocol}
\begin{center}
\def\svgwidth{\textwidth}
\input{kerb-pkinit-as-rep-ARTIFACT.pdf_tex}
\end{center}
\end{frame}

\begin{frame}{PKINIT - protocol}
\begin{center}
\def\svgwidth{\textwidth}
\input{kerb-pkinit-as-exchange-post-ARTIFACT.pdf_tex}
\end{center}
\end{frame}

\begin{frame}{PKINIT in FreeIPA}
\protect\hypertarget{pkinit-in-freeipa}{}
\begin{itemize}
    \item Default: {\bf exact certificate match} only
    \item Optional: \textbf{\em certificate mapping rules}
        \begin{itemize}
            \item {\tt ipa certmaprule-add certmap \textbackslash{} \\
                ~~~~{-}{-}maprule "(fqdn=\{subject\_dns\_name\})"}
        \end{itemize}
    \item Client certs can be signed by internal or third-party CA
\end{itemize}
\end{frame}

\begin{frame}{PKINIT - user experience}
\protect\hypertarget{pkinit-ux}{}
\begin{itemize}
    \item CLI: {\tt kinit -X X509\_user\_identity=FILE:cert.pem,key.pem}
    \item SSSD can integrate with Linux login managers
        \begin{itemize}
            \item Nice UX for smartcards and 2FA
            \item Windows offers a similiar experience
        \end{itemize}
\end{itemize}
\end{frame}

\begin{frame}[plain]
\centering
\huge Demo
\end{frame}


\begin{frame}{PKINIT - advantages}
\protect\hypertarget{pkinit-advantages}{}
\begin{itemize}
    \item No more passwords / client shared secret
    \item Key can reside on smart-card (e.g. Yubikey) / TPM / HSM
    \item The rest of the protocol is unchanged
\end{itemize}
\end{frame}

\begin{frame}{PKINIT - complexities}
\protect\hypertarget{pkinit-complexities}{}
\begin{itemize}
    \item X.509 PKI required
    \item Renewal considerations
    \item Revocation checking is hard (or time consuming)
    \item Hardware (e.g. smart cards) → additional cost
    \item Binding the public key to the principal - how?
\end{itemize}
\end{frame}

\begin{frame}[fragile]{PKINIT - key binding}
\protect\hypertarget{pkinit-key-binding-1}{}
\small
\begin{verbatim}
In addition to validating the client's signature, the KDC MUST also
check that the client's public key used to verify the client's
signature is bound to the client principal name specified in the AS-
REQ as follows:

1. If the KDC has its own binding between either the client's
   signature-verification public key or the client's certificate and
   the client's Kerberos principal name, it uses that binding.

2. Otherwise, if the client's X.509 certificate contains a Subject
   Alternative Name (SAN) extension carrying a KRB5PrincipalName
   (defined below) in the otherName field of the type GeneralName
   [RFC3280], it binds the client's X.509 certificate to that name.
\end{verbatim}
\end{frame}

\begin{frame}{PKINIT - key binding}
\protect\hypertarget{pkinit-key-binding-2}{}
\begin{itemize}
    \item Encode principal name in certificate ({\tt KRB5PrincipalName} SAN)
    \item Associate certificate/key with principal in database
        \begin{itemize}
            \item {\bf administrative overhead} due to renewal/rekey 
        \end{itemize}
    \item Other heuristics; for example:
        \begin{itemize}
            \item if certificate has {\tt dNSName} SAN,
                look for that host principal
            \item if certificate has {\tt rfc822Name} (email address) SAN,
                look for the corresonding user principal
        \end{itemize}
    \item {\bf Better not mess this up!}
\end{itemize}
\end{frame}



\begin{frame}[plain]
\centering
\huge CVE-2022-4254
\end{frame}


\begin{frame}{CVE-2022-4254 - what happened?}
\protect\hypertarget{vuln-desc}{}

\begin{itemize}
\item LDAP filter injection\footnote{CWE-90 - \url{https://cwe.mitre.org/data/definitions/90.html}}
\item FreeIPA {\bf not vulnerable in default configuration}
    \begin{itemize}
        \item only exact certificate match is enabled by default
    \end{itemize}
\item Bug introduced in SSSD v1.15.3 and resolved in v2.3.1
    \begin{itemize}
        \item Will also be fixed in of v1.16.6, if/when released
    \end{itemize}
\end{itemize}
\end{frame}

\begin{frame}[fragile]{CVE-2022-4254 - LDAP filter}
\protect\hypertarget{vuln-filter-wildcard}{}
\begin{lstlisting}[basicstyle=\ttfamily\footnotesize]
(&
  (|
    (objectClass=krbprincipalaux)
    (objectClass=krbprincipal)
    (objectClass=ipakrbprincipal)
  )
  (|
    (ipaKrbPrincipalAlias=(*\colorbox{yellow}{host/rhel78-0.ipa.test@IPA.TEST}*))
    (krbPrincipalName:caseIgnoreIA5Match:=host/rhel78-0.ipa.test@IPA.TEST)
  )
  ((*\colorbox{yellow}{fqdn=*.ipa.test}*))
)
\end{lstlisting}
\end{frame}

\begin{frame}[plain]
\centering
\Large What if your email address is \\
    ~\\
    \large
    \texttt{"bogus)(uid=admin)(cn="@ipa.test}
\end{frame}

\begin{frame}[plain]
\centering
\Large \ldots{}and your certmap rule looks like \\
    ~\\
    \large
    \texttt{(|(mail=\{subject\_rfc822\_name\})(entryDN=\{subject\_dn\})}
\end{frame}

\begin{frame}[fragile]{CVE-2022-4254 - domain takeover}
\protect\hypertarget{vuln-filter-mail}{}
\begin{lstlisting}[basicstyle=\ttfamily\footnotesize]
(&
  (|
    (objectClass=krbprincipalaux)
    (objectClass=krbprincipal)
    (objectClass=ipakrbprincipal)
  )
  (|
    (ipaKrbPrincipalAlias=(*\colorbox{yellow}{admin@IPA.TEST}*))
    (krbPrincipalName:caseIgnoreIA5Match:=admin@IPA.TEST)
  )
  (*\colorbox{yellow}{(|}*)
    (mail="bogus)
    ((*\colorbox{yellow}{uid=admin}*))
    (cn="@ipa.test)
    (entrydn=CN=alice,O=IPA.TEST 202211171708)
  )
\end{lstlisting}
\end{frame}

\begin{frame}{CVE-2022-4254 - mitigations}
\protect\hypertarget{vuln-mitigations}{}

\begin{itemize}
\item Upgrade to fixed/patched releases of SSSD
\item {\tt and-list} rules are harder to exploit than {\tt or-list}
\item Audit what data get included in certs, and how
\item Use exact certificate matching
\end{itemize}
\end{frame}

\begin{frame}{PKINIT - security considerations}
\protect\hypertarget{pkinit-security-considerations}{}

\begin{itemize}
\item Properly escape/sanitise all inputs, always.
\item Review CA trust, profiles, and validation behaviour
    \begin{itemize}
    \item Which CAs do you trust?  Who can issue certificates?
    \item How do you validate the data that go into a certificate?
    \item Can attributes be influenced by users or other parties?
    \end{itemize}
\item Just because a value is {\em valid} does not mean it's {\em benign}
\item Key/principal binding is a {\bf critical} aspect of PKINIT security
\end{itemize}
\end{frame}


\begin{frame}{Links}
\protect\hypertarget{links}{}

\begin{itemize}
\item Writeup: \url{https://is.gd/CVE_2022_4254}
\item Red Hat CVE database: \url{https://access.redhat.com/security/cve/CVE-2022-4254}
\end{itemize}

\end{frame}


\begin{frame}[plain]
\begin{columns}

  \begin{column}{.6\textwidth}

    \setlength{\parskip}{.5em}

    { \centering

    \input{cc-by-ARTIFACT.pdf_tex}

    \copyright~2023  Red Hat, Inc.

    { \scriptsize
    Except where otherwise noted this work is licensed under
    }
    { \footnotesize
    \textbf{http://creativecommons.org/licenses/by/4.0/}
    }

    }

    \begin{description}
      \item[Slides] \href{https://speakerdeck.com/frasertweedale}{speakerdeck.com/frasertweedale}
      \item[Blog] \href{https://frasertweedale.github.io/blog-redhat/}{frasertweedale.github.io/blog-redhat}
      \item[Email] \texttt{ftweedal@redhat.com}
      \item[Fediverse] \href{https://functional.cafe/@hackuador}{@hackuador@functional.cafe}
    \end{description}
  \end{column}

\end{columns}
\end{frame}

\end{document}
